Predicting disease traits based on genomics profiles plays an important role in personalized health care. With the advances in technologies, namely next generation sequencing, huge amount of genomics data are available, from genetic variations, transcriptomes, to epigenomics. Extensive researches in genomics have produced comparable amount of annotation databases, meta-analysis summary statistics. The explosion of genomics information has made genomics-based predictive modeling extremely appealing, at the mean time, challenging in terms of integrating multiple types of genomics data.  As different types of genomics information plays different role in disease mechanisms, traditional one model with one type of data only catches part of the big picture. Therefore, the purpose of the thesis is to develop modeling techniques that can integrate the diverse genomics information, to capture their interplay, collaboration in disease mechanisms, and hence achieving improved prediction performance. 

In chapter 1, we introduce each type of genomics data, and their by-product, annotations and summary statistics. The annotations and summary statistics reveal the underlying characteristics of genmoic features, i.e, the features of genomic features, we call them meta-features. Then we introduce existing predictive modeling techniques. The high dimensionality of genomics data in current situation is a major consideration in modeling process. The best tool up to now is regularized regression, and it is the one tool we are going to use for prediction. Detailed information about ridge regression, the lasso, the elastic net, and selected nonconvex regularization are provided. Apart from prediction, translating model results into actionable insights is of great interest. Knowing the set of genomic features contribute the most predictive power to the outcome can deepen the understanding of disease course, treatment effects. We look at the ``feature selection'' property realized by sparse regularized regression.

Two predictive models integrating genomics data with respect to survival outcomes are proposed in chapter 2 and 3. In chapter 2, we present a regularized hierarchical model. This model utilized hierarchical structure to integrate genomic meta-feature data. Mathematically, the method integrates meta-features linearly, modeling regression coefficients through the mean. The method developed in chapter 3 allows differential regularization parameters for each feature, guided by meta-features. This is a non-linear integration of data, modeling meta-features through the variance. We provided two options to integrate genomics data, as different data set has different dynamics. In simulation, we showed with informative meta-features, prediction performance improves considerably, and the quality of feature selection also improves for the second method. The proposed models are also applied to 2 genomics study data, gene expression signatures for breast cancer survival, anti-pd1 immunotherapy predictive biomarker for melonoma survival. 

Perspectives about data integration, high dimensionality, feature selection, and statistical inference are discussed in chapter 4. We also proposed potential future work about the topics.

\bigskip
\textbf{Keywords:} genomics; data integration; regularized regression; feature selection; high dimensional data; convex and nonconvex optimization 