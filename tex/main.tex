% USC Dissertation/Thesis LaTeX Template
% Edited by Ruda Zhang, 2020-10-08.
% -----------------------------------------------------------------------------
%	PACKAGES AND DOCUMENT CONFIGURATION
%-----------------------------------------------------------------------------

% Use `report` class with `USCthesis` package (style file) by Brian P. Gerkey
% Font size should be 11 or 12 points for regular paragraph text.
\documentclass[dissertation,12pt]{report}
% [options] can be any of these default/alternative flag:
%   dissertation/thesis, final/proposal, copyright/nocopyright,
%   fussy/sloppy, flushbottom/raggedbottom, clref/opref.
\usepackage[dissertation]{USCthesis}

% Packages required by `USCthesis.sty`.
\usepackage{hyperref}
\usepackage{setspace}
\usepackage{tabularx}
\usepackage{multirow}
\usepackage{multicol}
\usepackage{booktabs}
\usepackage{listings}
\usepackage{adjustbox}

% Line spacing and margins in compliance with USC graduate school guidelines.
\usepackage[margin=1in,footskip=.5in]{geometry}
\doublespacing

% Optional packages: mathematical fonts and symbols
% You may comment this section out if you don't need them.
\usepackage[T1]{fontenc} % font encoding for xelatex
\usepackage{amssymb}
\let\Bbbk\relax
\usepackage{newtxtext, newtxmath} 
\usepackage{mathtools}
\usepackage{amsmath}
\DeclareMathOperator*{\argmin}{argmin}
\usepackage{bm}
\usepackage{enumerate}

% Optional packages: graphics
\usepackage{graphicx}
\usepackage{float}
% You can use the "demo" option while editing to avoid compiling figures.
% \usepackage[demo]{graphicx}
% You can add absolute paths as well.
\graphicspath{{./}{../}{figures/}{../figures/}}

% Optional packages: bibliography with BibLaTeX.
% Comment this section out if you prefer BibTeX.
%\usepackage[
%    style=nature,
%    sorting=none,
%    isbn=false,
%    url=false,
%    doi=true,
%    eprint=false,
%    date=year,
%    maxnames=6,
%    minnames=6
%]{biblatex}
%\AtEveryBibitem{\clearfield{eventtitle}} 
%\AtEveryCitekey{\clearfield{eventtitle}}
%\AtEveryBibitem{\clearfield{pagetotal}} 
%\AtEveryCitekey{\clearfield{pagetotal}}
%\addbibresource{references.bib}

\usepackage{natbib}
\bibliographystyle{unsrtnat}
\setcitestyle{}
% Filler text for formatting. Comment these lines out for real writing.
%\usepackage[english]{babel}
%\usepackage{blindtext}


\begin{document}

%-----------------------------------------------------------------------------
%	TITLE PAGE
%-----------------------------------------------------------------------------

% Volume name could be added as option, e.g. `[Volume I]`.
\title{\textbf{\Large{Prediction and Feature Selection with Regularized Regression in Integrative Genomics}}}

\author{Dixin Shen}

% Committee list is only shown in `proposal` layout.
\committee{A.~Adams & (Chair)\\*
           B.~Bell\\*
           C.~Clausius\\*
           D.~Dirichlet\\*
           E.~Emory & (Outside Member)}

% Submission information is only shown in `final` layout.
\majorfield{BIOSTATISTICS}
\submitdate{August 2021}  % Must be one of the three dates allowed in the guideline

% Make sure everything, specially your title page, exactly follows the guideline:
% https://graduateschool.usc.edu/wp-content/themes/fictional-university-theme/assets/doc/Manuscript_Formatting_and_Documentation_Styles.pdf

%-----------------------------------------------------------------------------
%	PREFACE
%-----------------------------------------------------------------------------

% The preface environment prints the title page.
\begin{preface}

  % Dedication Page, which is truly unnecessary.
  %\prefacesection[Dedication]{}
  %\input{dedication}

  % Acknowledgement Page, which is also unnecessary for proposals.
  \prefacesection{Acknowledgements}
  I would like to express my greatest gratitude to my advisor and chair of my committee, Dr. Juan Pablo Lewinger for his guidance and support throughout my Ph.D. Your wisdom helped me become a better researcher, better person. I would also like to thank my committee members, Dr. David Conti, Dr. Duncan Thomas, Dr. Meredith Franklin, and Dr. Joseph Hacia for the insights and advice. It was a great pleasure to work with you. 

I would like to thank all my friends during my time at USC, who shared many great moments with me.

Last but not least, my family especially my parents, thank you for your unconditional support and encouragement.


  \tableofcontents
  \listoftables   % Comment this out if you don't have tables
  \listoffigures

  % Abstract Page
  \prefacesection{Abstract}
  Predicting disease traits based on genomics profiles plays an important role in personalized health care. With the advances in technologies, namely next generation sequencing, huge amount of genomics data are available, from genetic variations, transcriptomes, to epigenomics. Extensive researches in genomics have produced comparable amount of annotation databases, meta-analysis summary statistics. The explosion of genomics information has made genomics-based predictive modeling extremely appealing, at the mean time, challenging in terms of integrating multiple types of genomics data.  As different types of genomics information plays different role in disease mechanisms, traditional one model with one type of data only catches part of the big picture. Therefore, the purpose of the thesis is to develop modeling techniques that can integrate the diverse genomics information, to capture their interplay, collaboration in disease mechanisms, and hence achieving improved prediction performance. 

In chapter 1, we introduce each type of genomics data, and their by-product, annotations and summary statistics. The annotations and summary statistics reveal the underlying characteristics of genmoic features, i.e, the features of genomic features, we call them meta-features. Then we introduce existing predictive modeling techniques. The high dimensionality of genomics data in current situation is a major consideration in modeling process. The best tool up to now is regularized regression, and it is the one tool we are going to use for prediction. Detailed information about ridge regression, the lasso, the elastic net, and selected nonconvex regularization are provided. Apart from prediction, translating model results into actionable insights is of great interest. Knowing the set of genomic features contribute the most predictive power to the outcome can deepen the understanding of disease course, treatment effects. We look at the ``feature selection'' property realized by sparse regularized regression.

Two predictive models integrating genomics data with respect to survival outcomes are proposed in chapter 2 and 3. In chapter 2, we present a regularized hierarchical model. This model utilized hierarchical structure to integrate genomic meta-feature data. Mathematically, the method integrates meta-features linearly, modeling regression coefficients through the mean. The method developed in chapter 3 allows differential regularization parameters for each feature, guided by meta-features. This is a non-linear integration of data, modeling meta-features through the variance. We provided two options to integrate genomics data, as different data set has different dynamics. In simulation, we showed with informative meta-features, prediction performance improves considerably, and the quality of feature selection also improves for the second method. The proposed models are also applied to 2 genomics study data, gene expression signatures for breast cancer survival, anti-pd1 immunotherapy predictive biomarker for melonoma survival. 

Perspectives about data integration, high dimensionality, feature selection, and statistical inference are discussed in chapter 4. We also proposed potential future work about the topics.

\bigskip
\textbf{Keywords:} genomics; data integration; regularized regression; feature selection; high dimensional data; convex and nonconvex optimization 
\end{preface}

%-----------------------------------------------------------------------------
%	CONTENT STRUCTURE
%-----------------------------------------------------------------------------

% Better to separate LaTeX structure and content
\chapter{Introduction}
\label{cha:introduction}

\section{Background}
\label{sec:Background}
Technological advances of the recent decades such as microarrays and next generation sequencing (NGS) have revolutionized our ability to interrogate the structure, function, and regulation of the genome at an unprecedented scale. It has accelerated the investigation of the effects of the genome, epigenome and transcriptome on human health, and opened the door to personalized medicine. 

Personalized medicine refers to our ability to diagnose and predict health outcomes and response to treatment based on an individual's `omic' profile, so that treatment can be taylored to optimize the individual's health outcomes. 

On top of genomic, transcriptomic and epigenomic  profiles measured on individuals, there sits a layer of `meta knowledge' about gene function, gene products and gene regulation, that is critical for interpreting the results of studies of genomics and disease. These `meta-features' can include functional annotations and previous studies. 

The purpose of this thesis is to develop high-dimensional statistical methods to integrate  genomics data with meta-features into the predictive modeling process, with the ultimate goal of improving  prediction performance and producing more interpretable results.

\section{Genomics-based prediction of disease traits}
\label{sec:Prediction}
Among the many uses of prediction of disease outcomes based on genomic data, we can distinguish between two important applications: 1) disease prognosis, namely prediction of the course of disease, and survival. We refer to genomic features predictive of these outcomes as prognostic biomarkers; 2) prediction of treatment effect, based on patients' genomics profiles, i.e., how patients they respond to a treatment. We refer to genomic features predictive of these outcomes as predictive biomarkers \citep{mandrekar2010predictive}. 

In this thesis, the focus is on predictive modeling in cancer genomics, as array- and sequencing-based assays of tumor tissue has enabled clinically actionable insights for many types of cancers. We give a brief overview of some of the predictive modeling techniques commonly used. 


\subsection{Regression}
Regression methods are widely used techniques for constructing predictive models. The most popular are linear regression for quantitative traits, logistic regression for binary and multi-category traits, and Cox's proportional hazards regression for survival traits. These regression methods rely on linear combinations of the genomic features/predictors to predict the outcome of interest. The training data consists of a response vector $\bm{Y}$ of length $n$ (for survival outcomes, it is a response matrix of dimension $n \times 2$, survival time and censoring status) and a data/design matrix $\bm{X}$ of dimension $n \times p$, where $n$ is the number of independent samples, $p$ is the number of features. The linear predictors for each of the $n$ samples are defined as $\bm{\beta}^T\bm{x}_i$, where $\bm{x}_i$ is the features for the $i^{th}$ sample, $\bm{\beta}$ is the regression coefficients to be estimated. The optimization problems for the linear and logistic cases are shown below:
\begin{align}
    &\min_{\bm{\beta}} \left\{\frac{1}{2n} \sum_{i=1}^{n} (y_i-\bm{\beta}^T\bm{x}_i)^2 \right\}, \qquad\qquad\qquad\quad\;\: \text{linear regression} \label{eq1.1} \\
    &\min_{\bm{\beta}} \left\{-\frac{1}{n} \sum_{i=1}^{n} \left[y_i\bm{\beta}^T\bm{x}_i-\ln(1+e^{\bm{\beta}^T\bm{x}_i})\right]\right\}, \qquad \text{logistic regression} \label{eq1.2}
\end{align}

We will deal in detail with the Cox's proportional hazards regression in later sections, as survival is the main type of trait this thesis will be concerned  with. Incorporation of non-linearities are possible with regression, by the inclusion of higher order and interaction terms.

There are several requirements to apply standard regression methods. First, the number of features $p$ should be less than the number of samples $n$ for the model to be able to fit. However, this is usually not the case in genomics studies because of the large number of features examined and the limits to sample size due to costs and logistics. For example, biopsies could be highly invasive depending on the location of tumors, and hence reserved for some patients. As a result, the number of samples is small to moderate at best, typical number ranging from hundreds to thousands. By contrast, the number of genomic features is often huge. There may be tens (e.g., gene expression) or even hundreds of thousands features (e.g., methylation). If there are multiple omic types, the numbers can be even larger. Therefore, genomics data are typically very high dimensional ($p>>n$). Multicollinearity between features is another serious concern. Highly correlated features should have similar regression coefficients (negatively correlated markers have opposite sign, but similar in magnitude). But in standard regression, the coefficient estimates are highly unstable when the features are highly correlated. Other concerns include marker-marker interactions, missing data. 

Regression models cannot be fit with high-dimensional data. Therefore, regularization need to be introduced. Considering a linear regression, equation \eqref{eq1.1}, the solution to it is the ordinary least square (OLS), $\hat{\bm{\beta}}=(\bm{X}^T\bm{X})^{-1}\bm{X}^T\bm{Y}$. If $\bm{X}_{n\times p}$ is high-dimensional, $p>n$, the highest rank of matrix $(\bm{X}^T\bm{X})_{p\times p}$ is $\min(p,n)=n$, so it is a singular matrix. Mathematically, there is no solution to $\bm{\beta}$. Intuitively, the model is too complex to fit because there is not enough data. Regularization is a technique to control model complexity, by shrinking the regression coefficients. Regularized regression can be written as an optimization problem of the form:
\begin{equation}
    \min_{\bm{\beta}} \left\{-\ell(\bm{\beta}) + \lambda P(\bm{\beta})\right\}, \label{eq1.3}
\end{equation}
where $\ell(\bm{\beta})$ is the log of likelihood function. $P(\bm{\beta})$ is the regularization/penalty function, which penalizes the regression coefficients and shrinks their estimates toward zero. Shrinkage stabilizes the coefficient estimates (reduces model variance) and decreases model complexity (increases bias). Thus, by controlling the amount of shrinkage, the hyperparameter $\lambda \geq0$ controls the trade-off between model complexity (bias) and model stability (variance). Different types of regularization techniques will be discussed in section \ref{sec:pen_reg}.


\subsection{Other machine learning methods}
\label{sec:tree}
Diagnosis and prognosis of disease outcomes based on omics features are supervised classification problems in the machine learning parlance. Therefore, machine learning methods such as tree-based models, neural networks can be applied to these tasks. Classification and regression trees (CART) are known to excel at capturing complex interaction patterns. With multiple tree splits at different nodes of different features, tree-based methods are, at their essence multi-way interactions models. Omic features can contribute to prediction of disease outcomes in very different ways. Classical regression models assume additive contributions of the features but cannot handle multi-way interactions parsimoniously. Tree-based methods are a great complement to linear models in omic-based predictions. Gradient boosted tree methods are ensembles of many simple trees with only a few terminal nodes. While simple trees can often achieve only slightly better than random predictions, ensemble tree methods can often provide state-of-the-art prediction performance. We briefly describe`xgboost', one of the most widely used tree boosting algorithms \citep{chen2016xgboost}. `xgboost' has the objective function given by:
\begin{displaymath}
\text{obj}(\theta) = \sum_{i=1}^n l(y_i, \hat{y}_i) + \sum_{k=1}^K \Omega(f_k), 
\end{displaymath}
\begin{displaymath}
\hat{y}_i = \sum_{k=1}^Kf_k(x_i).
\end{displaymath}
$l(y_i, \hat{y}_i)$ is the loss function: a measure of `distance', for sample $i$, between its label $y_i$ and the model prediction $\hat{y}_i$. The prediction function $\hat{y}_i$ is an ensemble of a series of simple trees (weak learners), $\sum_{k=1}^Kf_k(x_i)$. The objective function $\text{obj}(\theta)$ is to be minimized, with the added regularization $\Omega(f_k)$ to control the model complexity. The optimization algorithm involves greedily optimizing one tree at a time with a gradient descent method. The `xgboost' style gradient tree boosting has enjoyed great success in many machine learning applications such as the Netflix prize challenge, and a number of Kaggle challenges. Because of its tree-based nature to explore interaction patterns, it is a good alternative to linear models.

\subsection{Comparison of predictive methods in genomics}
Gradient boosting machine and neural networks are often superior when the sample size is large, since there is enough information for them to explore complex non-linear patterns. But in scenarios with smaller sample sizes, regression approaches can often better capture overall feature effects in high dimension settings. This is the reason why regression methods are still the most widely used model in genomics, instead of tree-based methods and neural networks, despite their huge success in other areas.


\section{Regularized regression} \label{sec:pen_reg}
Regularization is essential to deal with high dimensional genomics data. There are many type of penalty functions. We describe below some of the most common.

\subsection{Ridge regression}
Ridge regression was proposed by \cite{hoerl1970ridge}. It puts a regularization on magnitude of regression coefficients, namely the squared size. Ridge regression is written as the following optimization problem 
\begin{equation}
    \min_{\bm{\beta}} \left\{ -\ell(\bm{\beta})+\lambda\|\bm{\beta}\|_2^2  \right\}. \label{eq1.4}
\end{equation}
$\lambda$ is the hyperparameter controlling the amount of regularization, the greater $\lambda$ is, the greater the amount of regularization on coefficients $\bm{\beta}$. The coefficients are shrunk toward zero. An equivalent way to write the ridge problem is 
\begin{equation}
    \begin{aligned}
    &\min_{\bm{\beta}} -\ell(\bm{\beta}), \\
    &\text{subject to} \qquad \|\bm{\beta}\|_2^2 \leq t, \label{eq1.5}
    \end{aligned}
\end{equation}
which is constrained optimization form of the above Lagrangian form \eqref{eq1.4}. The coefficients are restricted within a circle with diameter $t$, the $L_2$ norm ball. There is a one-to-one correspondence between the parameters $\lambda$ and $t$. When there are many correlated features in a standard regression model, their coefficients can be unstable due to high variance. By imposing a size constraint, the problem is alleviated. 

The solution to the ridge optimization problem is $\hat{\bm{\beta}}^{ridge}=(\bm{X}^T\bm{X}+\lambda \bm{I})^{-1}\bm{X}^T\bm{Y}$. Like the OLS solution, ridge regression solution is also a linear function of outcome $\bm{Y}$. It adds a positive constant $\lambda$ to the diagonal of $\bm{X}^T\bm{X}$ before inversion, making the matrix nonsingular even if $\bm{X}^T\bm{X}$ is not of full rank (high-dimensional setting). This is how ridge regression fit high-dimensional data and other ill-formed design matrix $\bm{X}$. In the case of orthonormal column spaces of $\bm{X}$, the ridge solution becomes a scaled version of the OLS solution, i.e., $\hat{\bm{\beta}}^{ridge}=\frac{1}{1+\lambda}\hat{\bm{\beta}}^{OLS}$, shrinking the coefficients by a fraction of $1+\lambda$. If the column spaces of $\bm{X}$ are not othonormal, ridge regression shrinks the directions with smallest variances the most. Those directions are in fact the principal components directions of $\bm{X}$. Principle components are linear combinations of the columns of $\bm{X}$. The first principle component has the largest sample variance (eigen value). Subsequent principal components have maximum variance subject to being orthogonal to the earlier ones. Hence, the small eigen value principle components directions are shrunk the most. 

Ridge regression has a Bayesian interpretation, assuming linear regression:
\begin{align*}
    &\bm{Y}|\bm{\beta};\bm{X} \sim N(\bm{X\beta}, \sigma^2\bm{I}), \\
    &\pi(\bm{\beta}) \sim N(0, \gamma^2\bm{I}).
\end{align*}
Both likelihood and prior are normal, therefore, the posterior distribution is also a normal. Because normal is its own conjugate family. The negative log posterior density of $\bm{\beta}$ is equal to the expression in equation \eqref{eq1.4}, with $\lambda=\sigma^2/\gamma^2$. Thus the ridge estimate is the mean and mode of the posterior distribution. In genomics, Bayesian ridge regression is the motivation of genomic best linear unbiased predictor (G-BLUP) \citep{de2013prediction}. 

Ridge regression shrinks coefficients toward zero, but never to exactly zero. In other words, it doesn't perform feature selection in terms of magnitude of regression coefficients. If coefficients shrink to zero, these features are no longer in the model, and thus not associated with outcome. Features with larger coefficients in magnitude, weather positive or negative, are strongly associated with outcome. However, ridge regularization is a widely used technique for controlling model complexity to balance the trade-off between bias and variance. The more complex the model, the less bias, but the larger variance, and vice versa. It is used in neural networks and gradient boosting machines, where it is known as weight decay.    

\subsection{Sparse regularized regression and feature selection} \label{sec:sparse}
\subsubsection{The Lasso}
Proposed by \citep{tibshirani1996regression}, the lasso is a regularization method like ridge, but performs feature selection. The lasso optimization problem, Lagrangian form, is defined as 
\begin{equation}
    \min_{\bm{\beta}} \left\{ -\ell(\bm{\beta})+\lambda\|\bm{\beta}\|_1  \right\}. \label{eq1.6}
\end{equation}
It can also be written in the equivalent constrained optimization problem just like ridge,
\begin{equation}
    \begin{aligned}
    &\min_{\bm{\beta}} -\ell(\bm{\beta}), \\
    &\text{subject to} \qquad \|\bm{\beta}\|_1 \leq t. \label{eq1.7}
    \end{aligned}
\end{equation}
The similarity to the ridge regression is the $L_2$ norm penalty function for ridge is replaced by $L_1$ norm penalty function for the lasso. The term sparse refers to a model with few nonzero coefficients. A key property of the lasso is its ability to yield sparse solutions. Lets look at the lasso estimator for linear regression. For the $j^{th}$ feature, i.e, the $j^{th}$ element of coefficients vector $\bm{\beta}$, the coordinate-wise update, for standardized features with mean 0 and variance 1, has the form
\begin{equation}
    \hat{\beta}_j^{lasso}=S(\frac{1}{n}\sum_{i=1}^{n}x_{ij}r_i^{(j)}, \lambda) \label{eq1.8}
\end{equation}
where 
\begin{itemize}
    \item $r_i^{(j)} = y_i-\sum_{l\neq j}x_{il}\hat{\beta}_l$ is the partial residual which removes from the outcome the current fit from all but the $j^{th}$ predictor. Because features are usually standardized to make the shrinkage comparable, $\frac{1}{n}\sum_{i=1}^{n}x_{ij}r_i^{(j)}$ is the simple least squares solution when fitting this partial residual to $x_{ij}$.
    \item $S(x, \lambda)$ is the soft-thresholding operator defined as 
    \begin{equation}
        \text{sign}(x)(|x|-\lambda)_+ = 
            \begin{cases}
                x-\lambda & \text{if $x>0$ and $\lambda<|x|$}\\
                x+\lambda & \text{if $x<0$ and $\lambda<|x|$}\\
                0 & \text{if $\lambda \geq |x|$}
            \end{cases} \label{eq1.9}      
    \end{equation}
\end{itemize}
One can derive the results using the notion of subgradients, the detailed derivation of coordinate descent are described in \cite{friedman2007pathwise}. Notice the lasso solution shrinks the regression coefficient (solution of least squares, $\frac{1}{n}\sum_{i=1}^{n}x_{ij}r_i^{(j)}$) by an amount of $\lambda$, as long as its magnitude/absolute value is larger than $\lambda$. For the features having smaller effect sizes than $\lambda$, they are shrunk to 0, thus being excluded to the model (Figure \ref{fig:soft_threshold}). This is the main difference between ridge and the lasso, while ridge regression does a proportional shrinkage, lasso translates each coefficient by a constant $\lambda$, truncating at zero. Therefore, the lasso has the ability to perform feature selection by excluding unimportant features. In this way, the lasso model is more parsimonious, more interpretable, compared to ridge, which keeps all the features in the model.
\begin{figure}[tbh]
  \centering
  \includegraphics[scale=0.6]{soft_threshold}
  \caption[Soft thresholding function $S(x, \lambda)=\text{sign}(x)(|x|-\lambda)_+$]{
    Soft thresholding function $S(x, \lambda)=\text{sign}(x)(|x|-\lambda)_+$. The figure is from \cite{hastie2019statistical}. The blue broken line is the soft threshoding estimator, along with the $45^{\circ}$ line in black.
  }
  \label{fig:soft_threshold}
\end{figure}

There are some important properties of the lasso in addition to feature selection.
\begin{itemize}
    \item Just like ridge regression, the lasso also has an Bayesian interpretation. The prior distribution of $\bm{\beta}$ is double exponential/Laplace for the lasso, instead of normal for ridge.
    \item Degrees of freedom: Suppose there are $p$ features, fitting a linear regression using only a subset of $k$ of these features, if these $k$ features were chosen without regard to the outcome, the procedure spends $k$ degrees of freedom. However, if the $k$ features were chosen using knowledge of the outcome, for example best subset selection, then the fitting procedure spends more than $k$ degrees of freedom. Such a fitting strategy is adaptive, as well as the lasso. The lasso, with a fixed penalty parameter $\lambda$, the number of nonzero coefficients $k_\lambda$ is an unbiased estimate of the degrees of freedom \citep{zou2007degrees, tibshirani2012degrees}. The reason that lasso has exactly $k$ degrees of freedom rather than larger than $k$ is that it not only selects features which inflates the degrees of freedom, but also shrinks the coefficients. 
    \item The number of nonzero coefficients is at most $n$, the sample size, when the data is high dimensional, $p>n$.
    \item Assume that the underlying true signal is sparse, the lasso recovers the true signals well. If the underlying truth is not sparse, the lasso does not work well. 
\end{itemize}

\subsubsection{The elastic net}
Proposed by \cite{zou2005regularization}, the elastic net combines the ridge and the lasso; it solves the convex optimization problem
\begin{equation}
    \min_{\beta} \left\{ -\ell(\bm{\beta})+\lambda\left[\frac{1}{2}(1-c)\|\bm{\beta}\|_2^2+c\|\bm{\beta}\|_1\right] \label{eq1.10} \right\} 
\end{equation}
where $c\in [0,1]$ is a parameter that controls whether the penalty function to be more close to lasso or more close to ridge. When $c=1$, it reduces to $L_1$ norm, lasso penalty; when $c=0$, it reduces to squared $L_2$ norm, ridge penalty. The coordinate-wise update for the elastic net linear regression, again assuming the features are standardized to mean 0 and variance 1, is 
\begin{equation}
    \hat{\beta}_j^{enet} = \frac{S(\frac{1}{n}\sum_{i=1}^{n}x_{ij}r_i^{(j)}, \lambda c)}{1+\lambda(1-c)} \label{eq1.11}
\end{equation}
We can see the elastic net estimator shrinks the regression coefficients in the way of both lasso and ridge: it has the soft-thresholding portion truncating at $\lambda c$; it also shrink the coefficients proportionally with a factor of $1+\lambda(1-c)$. Hence, the elastic net shrinks the coefficients and some of them to exactly 0, so feature selection.

\subsection{Discussion of ridge regression, the lasso, the elastic net, and best subset selection}
\label{comparison_reg}
Best subset selection finds for each $k\in\{0,1,2,\dots,p\}$ the subset of size $k$ that gives smallest residual sum of squares (validation error). Best subset selection linear regression is equivalent to $L_0$ constrained regression, when design matrix $\bm{X}$ is orthogonal:
\begin{equation}
\begin{aligned}
    &\min_{\bm{\beta}} \frac{1}{2n}\|\bm{Y}-\bm{X\beta}\|_2^2, \\
    &\text{subject to} \quad \|\bm{\beta}\|_0 \leq k, \label{eq1.12}
\end{aligned}
\end{equation}
where $\|\bm{\beta}\|_0=\sum_{j=1}^p I(\beta_j \neq 0)$, is defined as the number of nonzero coefficients. Strictly speaking, $L_0$ is not a norm because it does not have properties of a norm, but the naming and notation are widely used. The $L_0$ constraint penalizes the number of nonzero coefficients, instead of the magnitude of coefficients. This exactly describes the best subset selection setting. And because it does not shrink the coefficients, the $L_0$ estimates are unbiased, while other regularization estimates are biased toward 0. Although, best subset selection, or $L_0$ constrained regression is superior in coefficient estimation, feature selection, it does not have an efficient algorithm, when $\bm{X}$ is not orthogonal. If we want to select a best subset without knowing how many features should be included to be the best subset, $k$, then there are $2^p$ combinations of features need to be examined. In other words, there are no polynomial time algorithm to solve it; the problem is NP-hard. Many approximation methods have been proposed to solve the problem. Among them, iterative hard thresholding is well performed. The closed form hard thresholding solution for $L_0$ constrained linear regression when $\bm{X}$ is orthogonal is   
\begin{equation}
    \hat{\bm{\beta}}^{L_0}=H_{\sqrt{2\lambda}}(\frac{1}{n}\bm{X}^T\bm{Y}), \label{eq1.13}
\end{equation}
where $H_{\sqrt{2\lambda}}(\cdot)$ is the hard thresholding operator,
\begin{equation}
    H_{\sqrt{2\lambda}}(\frac{1}{n}\bm{X}^T\bm{Y})=
    \begin{cases}
        \frac{1}{n}\bm{X}^T\bm{Y} & \text{if $|\frac{1}{n}\bm{X}^T\bm{Y}|>\sqrt{2\lambda}$}, \\
        0 & \text{if $|\frac{1}{n}\bm{X}^T\bm{Y}|\leq\sqrt{2\lambda}$}.
    \end{cases}
\end{equation}
It does not shrink regression coefficients at all, but truncates at $\sqrt{2\lambda}$. This is in close relation to the soft thresholding estimator of lasso, equation \eqref{eq1.9}, which shrinks coefficients by the amount of $\lambda$ and truncates at $\lambda$. This is why hard thresholding is an unbiased estimator of regression coefficient. In fact, the lasso is one of many approximations to $L_0$ constrained problem. And it is the closest convex approximation to it, while $L_0$ is a nonconvex optimization problem. Figure \ref{fig:estimators} shows the estimators for best subset/$L_0$, ridge, and lasso in the case of orthonormal orthogonal $\bm{X}$. We can see the unbiased estimator of best subset; feature selection ability of best subset and lasso; and different shrinkage scheme between ridge and lasso.
\begin{figure}[tbh]
  \centering
  \includegraphics[scale=0.6]{estimator}
  \caption[Estimators of $\beta_j$ in the case of orthonormal columns of $\bm{X}$]{
    Estimators of $\beta_j$ for best subset, ridge, lasso, in the case of orthogonal column space of $\bm{X}$. Estimators are shown by broken red lines. The figure is from \cite{hastie2009elements}.
  }
  \label{fig:estimators}
\end{figure}

Ridge regression and the elastic net are also convex optimization problems. Since ridge, lasso and elastic net share this nice property, they have a highly efficient computational algorithm in pathwise coordinate descent \citep{friedman2007pathwise}. The algorithm solves the problems along a decreasing sequence of $\lambda$ values, for the purpose of tuning $\lambda$ via cross validation. Apart from giving a path of solutions, the algorithm exploits warm start, which initializes $\hat{\bm{\beta}}$ with the solutions of previous $\lambda$ value. This works because convex objective functions have continuous solutions along the path. By starting at previous solutions, coordinate descent updates need less iterations, thus leads to a more efficient and stable algorithm. On the other hand, iterative hard thresholding is not as efficient, and only guarantees to reach local minimums due to $L_0$'s nonconvexity. 

The lasso does not deal with highly correlated features very well; the solutions tend to be unstable. If there is a group of variables among which the pairwise correlations are very high, then the lasso tends to select only one variable randomly from the group. Ridge regression, on the other hand, shrink group correlated features toward each other. In other words, the features in a correlated group share the group effect evenly: having equal coefficients across the group and the coefficients add up to the group effect size. The elastic net is a combination of ridge and lasso. As we see the elastic net solution in equation \eqref{eq1.11}, it truncates the coefficients like lasso, shrink the coefficients proportionally like ridge. Therefore, it complements the inability of the lasso dealing with group correlated features with ridge regularization's grouping effect, which makes it a better choice when features are believed to have higher correlation structure.

\subsection{Nonconvex regularized regression} \label{sec:nonconvex}
Ridge regression, the lasso, the elastic net are all convex optimization problems. There are stable and efficient algorithms to solve it. And they always reach their global minimum solutions. Because of these, they are widely used for regularization, controlling model complexity. However, by moving from $L_2$ ridge to $L_1$ lasso, the shrinkage of some of the coefficients gets heavier, and finally being set to 0 when reach the lasso penalty. In fact, the lasso is the only convex regularization to produce sparse models. When the number of features is large and the true underlying model has only a few features, lasso is not able to shrink enough coefficients to 0. Nonconvex regularization leads to more sparse, less biased solutions. To see this, consider $L_q$ regularization,
\begin{equation}
    \min_{\bm{\beta}} \left\{ \frac{1}{2n}\|\bm{Y} - \bm{X\beta}\|_2^2 + \lambda \sum_{j=1}^{p}|\beta_j|^q \right\} \label{eq1.15}
\end{equation}
for $q\geq0$. It is the lasso for $q=1$, ridge for $q=2$. Figure \ref{fig:lq} displays $L_q$ regularization in the case of two inputs. For $0 \leq q <1$, the regularization is nonconvex, with the limiting $q=0$ corresponding to best subset selection. For these nonconvex constraints, they concentrate more mass in the coordinate directions, thus the solutions tend to be more sparse, and less shrinkage (biased toward 0). Unfortunately, along with nonconvexity comes combinatorial computational complexity and unstable algorithms. Alternative nonconvex regularization have been proposed.
\begin{figure}[tbh]
  \centering
  \includegraphics[width=\textwidth]{lq}
  \caption[Constraint regions $\sum_{j=1}^p|\beta_j|^q\leq1$ for different values of $q$] {
    Constraint regions $\sum_{j=1}^p|\beta_j|^q\leq1$ for different values of $q$. For $q<1$, the constraint region is nonconvex. The figure is from \cite{hastie2009elements}. 
  }
  \label{fig:lq}
\end{figure}

\subsubsection{Smoothed clipped absolute deviation penalty (SCAD) and minimax concave penalty (MCP)}
Proposed by \cite{fan2001variable}, the SCAD penalty defined on $[0, \infty)$ is given by (symmetric on $(-\infty, 0)$)
\begin{equation}
    P_{\lambda,\gamma}(\beta) = 
        \begin{cases}
            \lambda\beta & \text{if $\beta \leq \lambda$}\\
            \frac{\gamma\lambda\beta-0.5(\beta_2+\lambda^2)}{\gamma-1} & \text{if $\lambda<\beta\leq\gamma\lambda$}\\
            \frac{\lambda^2(\gamma^2-1)}{2(\gamma-1)} & \text{if $\beta>\gamma\lambda$}
        \end{cases} \label{eq1.16}      
\end{equation}
for $\lambda\geq0$ and $\gamma>2$. The univariate solution for a SCAD regularized simple linear regression coefficient is as follow
\begin{equation}
    \hat{\beta}=f_{SCAD}(z,\lambda,\gamma)= 
        \begin{cases}
            S(z,\lambda) & \text{if $|z|\leq 2\lambda$}\\
            \frac{S(z, \gamma\lambda/(\gamma-1))}{1-1/(\gamma-1)} & \text{if $2\lambda<|z|\leq\gamma\lambda$}\\
            z & \text{if $|z|>\gamma\lambda$}
        \end{cases} \label{eq1.17}      
\end{equation}
where $z=\frac{1}{n}\bm{X}^T\bm{Y}$ is the OLS solution.

Proposed by \cite{zhang2010nearly}, the MC+ penalty defined on $[0, \infty)$ is given by (symmetric on $(-\infty, 0)$)
\begin{equation}
    P_{\lambda,\gamma}(\beta) = 
        \begin{cases}
            \lambda\beta - \frac{\beta^2}{2\gamma} & \text{if $\beta \leq \gamma\lambda$}\\
            \frac{1}{2}\gamma\lambda^2 & \text{if $\beta>\gamma\lambda$}
        \end{cases} \label{eq1.18}      
\end{equation}
for $\lambda\geq0$ and $\gamma>1$. The univariate solution for a MC+ regularized simple linear regression coefficient is as follow
\begin{equation}
    \hat{\beta}=f_{MCP}(z,\lambda,\gamma)= 
        \begin{cases}
            \frac{S(z,\lambda)}{1-1/\gamma} & \text{if $|z|\leq \gamma\lambda$}\\
            z & \text{if $|z|>\gamma\lambda$}.
        \end{cases} \label{eq1.19}      
\end{equation}

The rational of SCAD and MCP is similar in that both penalties begin by applying same penalty as the lasso, and reduce the amount of penalty as the regression coefficient gets further away from zero. As a result of the penalty trend, when the coefficient is small in magnitude, it is shrunk to zero just like lasso; however, when the coefficient is large (larger than OLS solution), there is a transition region that shrinks the coefficient less than the lasso, and after the transition region, it is equal to the OLS solution without any shrinkage. This is a trend from less biased toward 0 to unbiased estimator, for those features with large effect sizes, thereby more likely to be associated with outcomes. Without the transition region, it is the hard thresholding estimator. The difference between SCAD and MCP is in the way they make transition. Figure \ref{fig:nonconvex_est} shows the trend of penalty functions of SCAD, MCP and their threshold functions. Indexed by nonconvexity parameter $\gamma$, it bridges the gap between lasso ($\gamma=\infty$) and best subset/hard threshold ($\gamma=2_+$ for SCAD, $1_+$ for MCP).   
\begin{figure}[tbh]
  \centering
  \includegraphics[width=\textwidth]{nonconvex_est}
  \caption[SCAD and MCP penalty functions and their corresponding threshold functions] {
    SCAD and MCP penalty functions and their corresponding threshold functions. Both are shown with $\lambda=1$ and different values for $\gamma$. The figure is from \cite{mazumder2011sparsenet}. 
  }
  \label{fig:nonconvex_est}
\end{figure}

The two nonconvex regularization techniques achieve less biased estimator, more sparse subset than the lasso. The choice of convex and nonconvex regularization depends on the application. For example, for a gene expression profile data, the underlying model is sparse but with a relatively large subset of features, the lasso is a better choice. Because with a large number of features in the model, say $1000-2000$, the direction of the feature coefficients are more meaningful rather than the magnitude of the coefficients. For a genetic association data, by which the underlying model only consists of a few markers, the accuracy of selection and unbiasedness of estimators are important. Hence, nonconvex regularization is a better suit in the situation.

We close the section by mentioning an approximation to $L_q$ ($0<q<1$) regularization that enjoys convex property.

\subsubsection{The adaptive lasso: approximation to nonconvex regularization}
Proposed by \cite{zou2006adaptive}, the adaptive lasso is a way of fitting models sparser than lasso. Using a pilot estimate $\tilde{\beta}$, the adaptive lasso has the form
\begin{equation}
    \min_{\bm{\beta}} \left\{ \frac{1}{2n}\|\bm{Y}-\bm{X\beta}\|_2^2+\lambda\sum_{j=1}^pw_j|\beta_j| \right\}, \label{eq1.20}
\end{equation}
where $w_j=1/|\tilde{\beta}_j|^v$. The adaptive lasso can be seen as an approximation to the $L_q$ regularization with $q=1-v$. We can see the adaptive lasso is convex in $\bm{\beta}$. Moreover, when the pilot estimates meet some regulatory conditions, the method recovers the true model under more general conditions than does the lasso. One can use least squares solution as the pilot estimate when $p<n$, univariate least squares solution when $p\geq n$. The indication is that, when least squares solution is small, the amount of penalty $w_j$ for feature $j$ is large, thereby $\hat{\beta}_j$ is more likely to be set to 0; when least squares solution is large, feature $j$ is penalized less (small $w_j$), then $\hat{\beta}_j$ is shrunk less, making it less likely to be 0 and less biased.

\section{Incorporating meta-features into modeling process}
As the types and the volume of genomics data are huge thanks to advanced high-throughput sequencing technologies, as well as the ever-growing annotation databases, there is increased need to integrate multiple types of genomics data, annotation data into modeling process. Because more related features provide more information to the outcome of interest, prediction performance can be improved. The traditional method is modeling one type of genomics data at a time, and combine the models in some form. As to utilizing annotation data, summary statistics, it is usually performed after modeling genomics data. This style of modeling different genomics data separately may ignore the interplay between them, the collective effect on the outcome. In this thesis, we introduced the concept meta-features, the features of the features. Now we introduce a second data matrix $\bm{Z}_{n\times p}$ that systematically stores extra data. Considering an annotation data that consists of several functional gene sets/pathways, and each gene set contains a group of genes. That is, if we have $p$ genomic features, $q$ functional gene sets (meta-features), the meta-feature matrix $\bm{Z}$ will have dimension $p\times q$, each row represents one genomic feature and has values of 0 or 1 indicating whether this genomic feature belongs to a gene set (1 indicates it belongs to the gene set, and 0 not). Table \ref{table:d1} shows the meta-feature matrix for gene sets.
\begin{table}[tbh]
    \centering
    \def\arraystretch{1.5}
    \begin{tabular}{|c|c|c|c|c}
        \hline
         & \textbf{gene set 1} & \textbf{gene set 2} & \textbf{gene set 3} & \dots \\ 
        \specialrule{.1em}{.05em}{.05em}
        gene 1 & 1 & 0 & 0 & \dots \\ 
        \hline
        gene 2 & 0 & 1 & 0 & \dots \\ 
        \hline
        gene 3 & 0 & 1 & 1 & \dots \\
        \hline
        \vdots & \vdots & \vdots & \vdots & $\ddots$ \\
    \end{tabular}
    \caption{Meta-feature matrix $\bm{Z}$ for gene sets/pathways}
    \label{table:d1}
    \end{table}

The usage of meta-feature matrix does not limit to annotation data, in fact, it can accommodate many types of information. We discuss 2 situations to show the flexibility of putting external data into meta-feature matrix $\bm{Z}$.

\begin{itemize}
    \item There are 3 types genomics data, gene expressions, single nucleotide polymorphisms (SNPs), DNA methylation  to be integrated into the modeling process. The meta-feature matrix tells which genomic feature is SNP, gene expression, or methylation. Table \ref{table:d2} shows the indicator meta-feature matrix. For example, ILMN\_343291 is a microarray probe, gene expression; rs10853372 is a SNP locus. 
    \begin{table}[tbh]
    \centering
    \def\arraystretch{1.5}
    \begin{tabular}{|c|c|c|c|}
        \hline
         & \textbf{Gene expression} & \textbf{SNP} & \textbf{Methylation} \\ 
        \specialrule{.1em}{.05em}{.05em}
        ILMN\_343291 & 1 & 0 & 0 \\ 
        \hline
        rs10853372 & 0 & 1 & 0 \\ 
        \hline
        ILMN\_1651210 & 1 & 0 & 0 \\
        \hline
        463100A3 & 0 & 0 & 1 \\
        \hline
        \vdots & \vdots & \vdots & \vdots \\
    \end{tabular}
    \caption{Meta-feature matrix $\bm{Z}$ for multiple types of genomics data}
    \label{table:d2}
    \end{table}
    
    \item There are summary statistics from similar studies on the same set of genomic features. These statistics from meta-analysis can be highly informative. They include p-values, hazard ratios, and source of features. In table \ref{table:d3}, gene BAX has a p\_value 0.0006 associated with the outcome, hazard ratio is 0.7605, the reason being included in the model is from previous GWAS studies. This is a hybrid matrix holding continuous values and indicator values: continuous values like p\-values, hazard ratios gives importance of the features; indicator variable tells the reason why the feature is included. 
    \begin{table}[tbh]
    \centering
    \def\arraystretch{1.5}
    \begin{tabular}{|c|c|c|c|c|c}
        \hline
         & \textbf{p\_value} & \textbf{Hazard ratio} & \textbf{Literature} & \textbf{GWAS}  \\ 
        \specialrule{.1em}{.05em}{.05em}
        BAX & 0.0006 & 0.7605 & 0 & 1 & \dots \\ 
        \hline
        IL6 & 0.2611 & -0.2077 & 1 & 0 & \dots \\ 
        \hline
        LDHB & $8.78\times 10^{-6}$ & 0.0768 & 0 & 1 & \dots \\
        \hline
        \vdots & \vdots & \vdots & \vdots & \vdots & $\ddots$ \\
    \end{tabular}
    \caption{Meta-feature matrix $\bm{Z}$ for summary statistics}
    \label{table:d3}
    \end{table}
\end{itemize}

With the above examples, we have shown the flexibility of the meta-feature matrix housing external information. Through the meta-feature matrix, we can integrate multiple types of genomics data, genomic annotation data, summary statistics from similar studies, and so on. It is the heart of our modeling process to integrate extra information that might be useful to prediction.

We propose two modeling strategy based on regularized regression. In chapter \ref{cha:xrnetcox}, we incorporate meta-features with a hierarchical structure
\[
\min_{\bm{\beta}} \left\{-\ell(\bm{\beta})+\frac{\lambda_1}{2}\|\bm{\beta}-\bm{Z\alpha}\|_2^2+\lambda_1\|\bm{\alpha}\|_1\right\},
\]
where $\ell(\bm{\beta})$ is log likelihood of regressing outcome on genomic features. And the coefficients of genomic features $\bm{\beta}$ are regressed on meta-features. This integrates meta-features linearly. In chapter \ref{cha:xtunecox}, we integrate meta-features in a non-linear way by allowing differential penalties for individual features, 
\begin{align*}
    &\min_{\bm{\beta}} \left\{-\ell(\bm{\beta}) + \sum_{j=1}^p \lambda_j\left[\frac{1}{2}(1-c)\beta_j^2 + c|\beta_j|\right]\right\}, \\
    &\lambda_j = e^{\bm{z_j}^T \bm{\alpha}}.
\end{align*}
The individualized penalty parameters $\lambda_j$'s are guided by meta-features, non-linearly with exponential function.


% Research Topic 1
\chapter{A Regularized Cox’s Hierarchical Model for Incorporating Annotation Information in Omics Studies}
\label{cha:xrnetcox}

\section{Abstract}
Associated with high-dimensional omics data there are often “meta-features” such as pathways and functional annotations that can be informative for predicting an outcome of interest. We extend to Cox regression the regularized hierarchical framework of Kawaguchi et al. (2021) for integrating meta-features, with the goal of improving prediction and feature selection performance with time-to-event outcomes. Regularization is applied to the omic features as well as the meta-features so that high-dimensional data can be handled at both levels. The proposed hierarchical Cox model can be efficiently fitted by a combination of iterative reweighted least squares and cyclic coordinate descent. In a simulation study we show that when the external meta-features are informative, the regularized hierarchical model can substantially improve prediction performance over standard regularized Cox regression. Importantly, when the external meta-features are uninformative, the prediction performance based on the regularized hierarchical model is on par with standard regularized Cox regression, indicating robustness of the framework. We illustrate the proposed model with applications to prediction of breast cancer survival, prediction of overall survival of melanoma, based on gene expression.

\section{Introduction}
Prediction based on high-dimensional omics data such as gene expression, methylation, and genotypes are important for developing better prognostic and diagnostic signatures of health outcomes. However, developing prediction models with high-dimensional omics data, where the number of features is often orders of magnitude larger than the available number of subjects is challenging. Sparse regularized regression methods, including the Lasso \citep{tibshirani1996regression} and its variants, elastic net \citep{zou2005regularization}, adaptive Lasso \citep{zou2006adaptive}, group Lasso \citep{yuan2006model} and others, control model complexity by shrinking all regression coefficients toward zero while setting some exactly to zero, effectively selecting features associated with the outcome. 

Kawaguchi et al. (2021) have shown that incorporating meta-features relating to the omics features can yield improved prediction of an outcome of interest and they developed a regularized hierarchical regression framework to incorporate external meta-feature information into the analysis of omics data. Example of meta-features are biological pathways containing different gene sets, functional information from databases like gene ontology, or results and summary statistics from previous studies. Their approach is implemented to handle quantitative and binary outcomes. The regularization type they applied to both features and meta-features is ridge only. Here we extend the regularized hierarchical model of Kawaguchi et al. (2021) to handle time to event outcomes and also add the lasso, elastic net to meta-features.

There are numerous approaches for testing enrichment of meta-features after an analysis relating the genomic features to an outcome of interest is performed. For example, \cite{subramanian2005gene} developed gene set enrichment analysis (GSEA) to yield insights on gene group level. The genes in a group, which is a meta-feature of the genes, share common chromosomal location or biological function. The enrichment analysis is performed after prior analysis of single gene differential expression analysis. However, there are few approaches capable of incorporating meta-features directly into the modeling process, rather than in a post-hoc fashion. Approaches to incorporate meta-features include the application of differential penalization based on external information and two-stage regression methods, where the outcome is first regressed on the genomic features and the resulting effect estimates are in turn regressed on the external meta features. \cite{tai2007incorporating} grouped genes based on existing biological knowledge and applied group-specific penalties to nearest shrunken centroids and penalized partial least squares. \cite{bergersen2011weighted} incorporates external meta-feature information by weighting the LASSO penalty of each genomic feature with some function of meta-feature. \cite{zeng2021incorporating} on the other hand, incorporates external meta-feature to customize the penalty of each feature with a different function of meta-feature. These three methods are based on idea 1), which no longer assuming every genomic feature are equally important, but of different importance based on external information. However, they cannot handle large number of meta-features. \cite{chen2007enriching} applied the idea of hierarchical modeling in a Bayesian framework, where second stage linear regression is normal prior distribution, first stage regression is normal conditional distribution, and estimate first stage regression coefficients with posterior estimator. This method does not apply to high dimensional data since it is standard regression with no regularization. The above data integration methods improve prediction compared to modeling with genomic features only. However, none of the approaches above can handle time-to-event outcomes.

In this chapter, we introduce a regularized Cox’s proportional hazard hierarchical model to integrate meta-features. We will see that when external meta-features are informative, regularized hierarchical modeling improves prediction performance considerably. On the other hand, we also show that when the external meta-features are not informative, it does not perform worse than standard regularized model, which does not use any external information. This shows that the model is robust to the informativeness of the meta-features and can be safely used when the meta-feature informativeness is a priori unknown, as it is typically the case. The model can be efficiently fitted using a combination of iterative reweighted least squares and cyclic coordinate descent as proposed for Lasso Cox regression by \cite{simon2011regularization}.

\section{Methods}
\subsection{Setup and notation}
We assume a survival analysis setting with outcome $(\bm{y},\bm{\delta})=(y_1,\dots,y_n,\delta_1,\dots,\delta_n)$, where $\bm{\delta}=(\delta_1,\dots,\delta_n)$ is a vector of censoring status for each subject, $\delta_i=1$ represents event happens, $\delta_i=0$ represents right-censoring; $\bm{y}=(y_1,\dots,y_n)$ is the vector of observed time, if $\delta_i=1$, $y_i$ is event time, and if $\delta_i=0$, $y_i$ is censoring time, for $n$ subjects. Let $\bm{X}$ denote the $n\times p$ design matrix, where $p$ is the number of features, each row represents the observations on one subject, and each column represents the values of one feature across the $n$ subjects. We are particularly interested in the high dimension setting, $p>>n$, where the number of features is larger than the sample size. The goal is to develop a predictive model for the outcome $(\bm{y},\bm{\delta})$ based on the data $\bm{X}$.

In a genomics context, the time-to-event outcome $(\bm{y},\bm{\delta})$ could be event free time, time to disease relapse, time to death. The design matrix $\bm{X}$ could be genotypes, gene expressions, DNA methylation. For example, in Molecular Taxonomy of Breast Cancer International Consortium (METABRIC) data, outcome $(\bm{y},\bm{\delta})$ is breast cancer specific survival, data matrix $\bm{X}$ represents gene expressions with dimension number of patients$\times$number of genes.

Associated with each feature there is typically a set of meta-features annotations. If $\bm{X}$ consists of gene expression values, pathway gene sets could be meta-features indicating the set of genes involved. As for the METABRIC example, 4 meta-features are believed to be associated with breast cancer: genes with mitotic chromosomal instability (CIN), mesenchymal transition (MES), lymphocyte-specific immune recruitment (LYM), and FGD3-SUSD3 genes. Each meta-feature consists a vector of indicator variables for whether a gene belongs to the functional gene group. The genomic meta-features can be collected into a matrix $\bm{Z}$ of dimensions $p\times q$, where $q$ is the number of meta-features. We propose a penalized hierarchical regression for integrating the external meta-feature information in $\bm{Z}$ for predicting time-to-event outcomes based on the features in $\bm{X}$
\begin{equation}
    \min_{\bm{\alpha},\bm{\beta}} \left\{ -\frac{1}{n}\ln{L_B(\bm{\beta})}+\frac{\lambda_1}{2}\|\bm{\beta}-\bm{Z\alpha}\|_2^2+\lambda_2\|\bm{\alpha}\|_1 \right\}, \label{eq2.1}
\end{equation}
where $L_B(\bm{\beta})$ is the negative log of the Cox partial likelihood function (see below), $\bm{\beta}$ is a length $p$ vector of regression coefficients corresponding to the $p$ features in $\bm{X}$, and $\bm{\alpha}$ is a length $q$ vector of regression coefficients corresponding to the meta-features. The objective function in \eqref{eq2.1} can be viewed as arising from a hierarchical model. In the first level of the hierarchy, the negative log partial likelihood $L_B(\bm{\beta})$ term in \eqref{eq2.1} corresponds to the time-to-event outcome modeled as a function of $\bm{X}$ via a Cox’s proportional hazard regression model. In the second level, the $L_2$ penalty term $\|\bm{\beta}-\bm{Z\alpha}\|_2^2$ corresponds to a linear regression of the estimate of $\bm{\beta}$ on the meta-feature information $\bm{Z}$. It can also be thought of as an $L_2$ regularization term that shrinks the estimate of $\bm{\beta}$ toward $\bm{Z\alpha}$ rather than to the usual shrinkage toward zero. In the third level of the hierarchy, the term $\|\bm{\alpha}\|_1$ is an $L_1$ regularization penalty on the vector of estimated efficts $\hat{\bm{\alpha}}$. It enables the selection of important meta-features by shrinking many of its components to $0$. The hyperparameters $\lambda_1,\lambda_2\geq0$ control the degree of shrinkage/regularization applied to each of the penalty terms and can be tuned by cross-validation. Finally, note that when $\bm{\alpha}=0$, the objective function \eqref{eq2.1} reduces to a standard $L_2$-regularized Cox regression.

The partial likelihood function $L_B(\bm{\beta})$ in \eqref{eq2.1} is the Breslow approximation \citep{breslow1972contribution} to the Cox partial likelihood. Letting $t_1<t_2<\cdots<t_k (k=1,2,\dots,D)$ be unique event times arranged on increasing order, the Cox model assumes proportional hazards: 
\begin{equation}
    h(t,\bm{x}_j)=h_0(t)\exp{(\bm{x}_j^T\bm{\beta})}, \label{eq2.2} 
\end{equation}
where $h(t,\bm{x}_j)$ is the hazard rate for subject $j$ with feature values $\bm{x}_j$ at time $t$; $h_0(t)$ is baseline hazard rate at time $t$, regardless of the feature values. The Cox partial likelihood function \citep{cox1972regression} can then be written as
\begin{equation}
    L(\bm{\beta})=\prod_k \frac{e^{\bm{x}_k^T\bm{\beta}}}{\sum_{j\in R_k}e^{\bm{x}_j^T\bm{\beta}}}, \label{eq2.3}
\end{equation}
where $R_k=\{j:y_j\geq t_k\}$, is the risk set at time $t_k$, i.e., the set of all subjects who have not experienced the event and are uncensored just prior to time $t_k$. The partial likelihood function allows estimation of $\bm{\beta}$ without explicitly modeling the baseline $h_0$, and it depends only on the order in which events occur but not on the exact times of occurrence. However, the partial likelihood assumes that event times are unique. To handle ties, where multiple individuals experience the event at the same time, we use the Breslow approximation of the partial likelihood in \eqref{eq2.3}
\begin{equation}
    L_B(\bm{\beta})=\prod_k \frac{\exp{(\sum_{j\in D_k}\bm{x}_k^T\bm{\beta})}}{(\sum_{j\in R_k}e^{\bm{x}_k^T\bm{\beta}})^{d_k}}, \label{eq2.4}
\end{equation}
where $D_k=\{j:\delta_j=1,y_j=t_k\}$, is the set of individuals who have event time $y_k$, and $d_k=\sum_jI(\delta_j=1,y_j=t_k)$ is the number of events at time $y_k$. Breslow’s likelihood function automatically reduces to the partial likelihood when there are no ties. 

\subsection{Model fitting}
The objective function \eqref{eq2.1} can be minimized efficiently using iterative reweighted least squares combined with coordinate descent \citep{simon2011regularization}. If the current estimates of the regression coefficients are $(\tilde{\bm{\beta}}, \tilde{\bm{\alpha}})$, we form a quadratic approximation to the negative log-partial likelihood by Taylor series around the current estimates. The approximated objective function has the form:
\begin{equation}
    \min_{\bm{\alpha},\bm{\beta}} \left\{ -\frac{1}{2n}(\bm{y}'-\bm{X\beta})^T\bm{W}(\bm{y}'-\bm{X\beta})+\frac{\lambda_1}{2}\|\bm{\beta}-\bm{Z\alpha}\|_2^2+\lambda_2\|\bm{\alpha}\|_1 \right\}, \label{eq2.5}
\end{equation}
where 
\begin{equation}
    \bm{y}'=\tilde{\bm{\eta}}+\bm{W}^{-1}(\bm{\delta}-\text{diag}[\exp{(\ln{H_0(\bm{y})}+\tilde{\bm{\eta}})}]), \label{eq2.6} 
\end{equation}
\begin{equation}
    \bm{W}=\text{diag}\left[\exp{(\ln{H_0(\bm{y})}+\tilde{\bm{\eta}})}\right] - \text{diag} [e^{\tilde{\bm{\eta}}}]\bm{M}\text{diag}\left[\frac{h_{0k}^2}{d_k}\right]\bm{M}^T\text{diag}[e^{\tilde{\bm{\eta}}}]. \label{eq2.7}
\end{equation}
In \eqref{eq2.6} and \eqref{eq2.7}, $\text{diag}[\bm{a}]$ is a diagonal matrix with vector $\bm{a}$ as diagonal elements. $\bm{M}$ is an $n\times D$ indicator matrix with $(i,k)^{th}$ element $I(y_i\geq t_k)$. Also, $\tilde{\bm{\eta}}=\bm{X}\tilde{\bm{\beta}}$ is the linear predictor; $h_{0k}=\frac{d_k}{\sum_{j\in R_k}\exp{(\tilde{\eta}_j)}}$ is estimated baseline hazard rate at event time $y_k$; $H_0(y_i)=\sum_{k:y_k\leq y_i}h_{0k}$ is cumulative baseline hazard at time $y_i$. In the first part of quadratic approximation \eqref{eq2.5}, $-\frac{1}{2n}(\bm{y}'-\bm{X\beta})^T\bm{W}(\bm{y}'-\bm{X\beta})$ can be viewed as a weighted version of least squares as $\bm{y}'$ work as responses, $\bm{W}$ as weights. Weight matrix $\bm{W}$ is usually a diagonal matrix, however, in Cox proportional hazard model, $\bm{W}$ is a full symmetric matrix as shown in \eqref{eq2.7}. This leads to computational difficulty as it requires calculation of $O(n^2)$ entries. According to \cite{simon2011regularization}, we can compute only the diagonal entries of $\bm{W}$ without much loss of accuracy, thereby speeding up implementation. The diagonal elements of $\bm{W}$, $w_i$ has the form:
\begin{equation}
    w_i=\sum_{k\in C_i}\frac{d_k e^{\tilde{\eta}_i}}{\sum_{j\in R_i}e^{\tilde{\eta}_j}}-\sum_{k\in C_i}\frac{d_k (e^{\tilde{\eta}_i})^2}{(\sum_{j\in R_i}e^{\tilde{\eta}_j})^2}, \label{eq2.8}
\end{equation}
where $C_i$ is the set of unique event time $t_k$ such that $t_k<y_i$ (the times for which observation $i$ is still at risk). In computing weights $w_i$’s, one bottleneck is that for each $k$ in $C_i$, we need to calculate $\sum_{j\in R_i}e^{\tilde{\eta}_j}$. Both $C_i$ and $R_k$ have $O(n)$ elements, so the weight computation is $O(n^2 )$ time. However, if we sort $y_i$’s in non-decreasing order, note that $\sum_{j\in R_i}e^{\tilde{\eta}_j}$ can be calculated in a cumulative summation fashion: for the risk set $R_{k+1}$, the only difference between it and $R_k$ are the observations that are in $R_k$ but not in $R_{k+1}$, ${j: t_k\leq y_j<t_{k+1}}$. This same idea can also be applied to calculate $\sum_{k\in C_{i+1}}\frac{d_k}{\sum_{j\in R_i}e^{\tilde{\eta}_j}}$. And the weight computation complexity can be reduced to $O(n)$. Details are in Appendix \ref{a.1}.

Now, let $\bm{\gamma}=\bm{\beta}-\bm{Z\alpha}$, and use only diagonal elements of $\bm{W}$, then \eqref{eq2.5} can be written as:
\begin{equation}
    \min_{\bm{\alpha},\bm{\beta}} \left\{ \frac{1}{2n} \sum_{i=1}^n w_i(y_i'-\bm{\gamma}^T\bm{x}_i-\bm{\alpha}^T(\bm{XZ})_i)^2+\frac{\lambda_1}{2}\|\bm{\gamma}\|_2^2+\lambda_2\|\bm{\alpha}\|_1 \right\}, \label{eq2.9}
\end{equation}
where $(\bm{XZ})_i$ is the $i^{th}$ row of $n\times q$ matrix $\bm{XZ}$. This reduced the problem to repeatedly solving the regularized, weighted least squares problem \eqref{eq2.9} using cyclic coordinate descent \citep{friedman2010regularization}. Details are given in Appendix \ref{a.2}.

\subsection{Two-dimensional hyperparameter tuning}
The optimization approach described above is for fitting the model for one combination of the tuning parameters $\lambda_1,\lambda_2$. More than one value combination of $\lambda_1,\lambda_2$ are usually of interest, as $\lambda_1,\lambda_2$ are tuned by cross-validation to get the best performance out of the model. For the proposed model, a two-dimensional grid of $\lambda_1,\lambda_2$ values are constructed, and pathwise coordinate optimization \citep{friedman2007pathwise} is applied along the two-dimensional path. The pathwise algorithm utilizes current estimates as warm start, since the solutions to the convex problem \eqref{eq2.9} is continuous. This character makes the algorithm remarkably efficient and stable.

In detail, for $\lambda_1$ of the the two-dimensional grid, which controls the amount of shrinkage to $L_2$ term $\|\bm{\beta}-\bm{Z\alpha}\|_2^2$, or in the transformation form $\|\bm{\gamma}\|_2^2$, since we initialize $\tilde{\bm{\gamma}}=0, \tilde{\bm{\alpha}}=0$, starting with $\lambda_{1\max}=1000\times \max_j\frac{1}{n}\sum_{i=1}^nw_i(0)x_{ij}y'_i(0)$ gives small value solutions to $\hat{\bm{\gamma}},\tilde{\bm{\alpha}}$, making the algorithm faster to converge. While for $\lambda_2$, which controls the amount of shrinkage to $L_1$ term $\|\bm{\alpha}\|_1$, $\lambda_{2\max}=\max_k \frac{1}{n}\sum_{i=1}^n w_i(0)(xz)_{ik}y'_i(0)$ is the smallest value that makes the entire vector $\hat{\bm{\alpha}}=0$. We compute the solutions for a decreasing sequence of $\lambda_1,\lambda_2$. More specifically, we start with $\lambda_{1\max},\lambda_{2\max}$, select $\lambda_{\min}=0.001\lambda_{\max}$, and construct a sequence of 20 $\lambda$ values from $
\lambda_{\max}$ to $\lambda_{\min}$ on log scale, therefore there are 400 $\lambda_1,\lambda_2$ combination of values in total. In order to apply warm start, we fix one value of $\lambda_1^{(m_1)}, 1\leq m_1\leq 20$, decrease $\lambda_2$ along the sequence, but keeping the solutions for $\lambda_{1}^{(m_1)}, \lambda_{2\max}$, so that when we work out the solutions for one sequence of $\lambda_2$ values and go to $\lambda_{1}^{(m_1+1)}, \lambda_{2\max}$, (where $\lambda_{1}^{(m_1+1)}$ is the next value along the sequence of $\lambda_1$), we can use the solutions of $\lambda_{1}^{(m_1)}, \lambda_{2\max}$ for warm start (figure \ref{fig:2d}).
\begin{figure}[tbh]
  \centering
  \includegraphics[scale=0.8]{2D-grid}
  \caption[Two-dimensional grid pathwise optimization]{
    Two-dimensional grid pathwise optimization.
  }
  \label{fig:2d}
\end{figure}

\subsection{Summary}
Summarizing procedures for fitting regularized hierarchical Cox model: 
\begin{enumerate}
    \item Initialize $\bm{\beta}$ and $\bm{\alpha}$ with $\widetilde{\bm{\beta}}$ and $\widetilde{\bm{\alpha}}$.
    \item For each $\lambda_1, \lambda_2$, repeat, until convergence of $(\hat{\bm{\beta}},\hat{\bm{\alpha}})$: 
    \begin{itemize}
        \item Compute weights $\bm{W}$ and working response $\bm{y}'$ with current estimate $(\tilde{\bm{\beta}},\tilde{\bm{\alpha}})$, form the quadratic approximation, equation \eqref{eq2.5}
        \item Find minimizer $(\hat{\bm{\beta}},\hat{\bm{\alpha}})$, solution to equation \eqref{eq2.9}, using coordinate descent
        \item Set $(\tilde{\bm{\beta}},\tilde{\bm{\alpha}})=(\hat{\bm{\beta}},\hat{\bm{\alpha}})$
    \end{itemize}
\end{enumerate}

\section{Simulations}
\subsection{Simulation methods}
We performed a simulation study to evaluate the predictive performance of the hierarchical Cox’s regression model compared to standard penalized Cox’s regression. The main parameters we control include informativeness of the meta-features, sample size, number of features and number of meta-features. We generated the $p\times q$ meta-feature matrix $\bm{Z}$, with each element drawn from an independent Bernoulli variable with probability $0.1$. This mimics binary indicators for whether a gene belongs to a particular biological pathway.  

The first level regression coefficients are generated as $\bm{\beta}=\bm{Z\alpha}+\bm{\varepsilon}$, where $\bm{\varepsilon} \sim N(0, \sigma^2\bm{I})$. To control the predictive power of the meta-features, we set the signal-to-noise ratio, $\text{SNR}=\bm{\alpha}^T\text{cov}(\bm{Z})\bm{\alpha}/\sigma^2$, where the signal is the variance of $\bm{\beta}$ explained by model $\bm{Z\alpha}$, and $\sigma^2$ is the noise. A higher signal-to-noise ratio implies a higher level of informativeness of the meta-features with respect to the coefficients $\bm{\beta}$. The data matrix $\bm{X}$ is generated by sampling from a multivariate normal distribution, $N(0,\Sigma)$, where the covariance matrix $\Sigma$ has an autoregressive correlation structure $\Sigma_{ij}=\rho^{|i-j|}$ for $i,j=1,\dots,p$.

The cumulative distribution function of the Cox proportional hazard model is given by
\begin{equation*}
    F(t|\bm{x})=1-\exp\left[-H_0(t)e^{\bm{\beta}^T\bm{x}}\right],
\end{equation*}
where $H_0(t)$ is baseline cumulative hazard function. Using the inverse probability integral transform \citep{bender2005generating}, we generated survival times $t$ as:
\begin{equation}
    t=H_0^{-1}\left[-\ln(U)e^{-\bm{\beta}^T\bm{x}}\right], \label{eq2.10}
\end{equation}
where $U\sim \text{uniform}[0,1]$. For the baseline hazards we used a Weibull distribution, which has cumulative hazard function $H_0(t)=(\frac{t}{b})^v$. The baseline Weibull parameters were set to $b=5,v=8$, which result in survival times in the range $0$ to $20$. We simulated the censoring time, $c$, based on an exponential distribution with density $f(c)=\exp(\lambda c)$, with $\lambda=0.06$. Then, the time-to-event outcome $(y_i,\delta_i)$ is generated as $(\min(t_i,c_i),I(t_i<c_i))$. The value of exponential distribution parameter $\lambda$ was chosen to result in a ratio of subjects experiencing the event vs. subjects experiencing censoring of about $2$ to $1$.

To control the predictivity of the features $\bm{X}$ for the outcome $\bm{y}$, we set Harrell’s concordance index (C-index) \citep{harrell1982evaluating} as the performance metric. It is defined as the probability that a randomly selected patient who experienced an event has a higher risk score $\bm{\beta}^T\bm{x}$ than a patient who has not experienced an event at a given time. The C-index is an analog of the area under the ROC for time-to-event data. The higher the C-index, the better the model can discriminate between subjects who experience the outcome of interest and subjects who do not or have not yet. To control the C-index we added random noise to the survival times $t$, where the noise is distributed as a normal with mean zero and a variance value set to yield a C-index of 0.8 across all simulation scenarios. This is the population/theoretical C-index of the generated survival data, achievable if $\bm{\beta}$ were known or if one had an infinite sample size. When $\bm{\beta}$ is estimated from a finite training set, the achieved model C-index will be lower.

We simulated a base case scenario with sample size $N=100$, number of features $p=200$, and number of meta-features $q=50$. This is a high dimensional setting, $p>>N$, typical of genomic studies. The first 20\% of the coordinates of the meta-feature level coefficients $\bm{\alpha}$ were set to be 0.2, and the rest set to 0. In the base scenario, the meta-features are highly informative, with a signal noise ratio set to 2. The covariance matrix $\bm{\Sigma}$ of $\bm{X}$ has autoregressive-1 structure, parameter $\rho=0.5$, so that the features are moderately correlated. In the following simulation situations, we vary one of the parameters and hold the others fixed. Simulations were performed with $B=100$ replicates for all scenarios. The models were trained on a training set of size $N$ (100 in the base scenario but varied in the experiments below), with the hyper-parameters $\lambda_1,\lambda_2$ tuned on an independent validation set of the same size as training set. The final predictive performance was evaluated on a large test set of size 10,000.

We run a series of experiment varying one key parameter at a time from the base case scenario as follows:

Experiment 1: varying the signal-to-noise ratio of the meta-features from completely uninformative, (SNR$=0$), to slightly informative (SNR$=0.1$), to moderately informative, (SNR$=0.8$), to highly informative (SNR$=2$).

Experiment 2: Varying the sample size from low to high, $N=100,200,500$.

Experiment 3: Varying the number of features from low to high: $p=200,500,1000$.

Experiment 4: Varying the number of meta-features from low to high: $q=20,50,100$.

\subsection{Simulation results}
The results of the experiments are shown in Figure \ref{fig:sim1}. In each panel, the horizontal dashed line representing the population/theoretical C-index, the maximum achievable with infinite training data, is provided as a reference for each parameter setting. We compared the performance of the hierarchical ridge-lasso Cox model incorporating meta-features to that of a standard ridge Cox model.

With informative meta-features (SNR$>0$ in experiments 1-4) the hierarchical ridge-lasso model consistently outperforms the standard ridge model, with the performance gain over the standard ridge model increasing with the informativeness of the meta-features (experiment 1). Importantly, when the meta-features are completely uninformative, the hierarchical ridge-lasso model performs only slightly worse than standard ridge model (experiment 1, SNR$=0$). This shows robustness of the hierarchical ridge-lasso to uninformative meta-features.

Experiment 2 shows that the gains in performance of the hierarchical ridge-lasso over the standard ridge model can be quite large, particularly when the sample size is small. As the sample size $N$ increases, the performance of both models increases and the difference between the two is reduced. 

As the dimensionality $p$ of the features increases (experiment 3), the performance of the standard ridge model deteriorates dramatically, while the performance of the hierarchical model only decreases slowly as the information in the meta-features helps stabilize its performance.

In experiment 4, the performance of the standard ridge model does not change, as it does not utilize meta-feature information. However, for the hierarchical ridge-lasso model, the performance decreases as the number of noise meta-features increases (the number of informative meta-feature is fixed at 10 and the additional meta-features are noise meta-features).  
\begin{figure}[tbh]
  \centering
  \includegraphics[width=\textwidth]{sim1}
  \caption[Simulation results (`xrnet'): prediction performance]{
    Simulation results: prediction performance.
  }
  \label{fig:sim1}
\end{figure}

We also examined the ability of the model to select informative meta-features by second-level Lasso penalty. In particular, we looked at the true and false positive meta-feature selection rate in experiment 1, where the second level meta-features informativeness varies (Figure \ref{fig:sim2}). We see that as the SNR of the meta-features increases, the true positive selection rate of informative meta-features improves dramatically (Figure 2.3a) at the cost of a slight increase in the false positive rate (Figure 2.3b).  
\begin{figure}[tbh]
  \centering
  \includegraphics[width=\textwidth]{sim2}
  \caption[Simulation results (`xrnet'): meta-feature selection]{
    Simulation results: meta-feature selection.
  }
  \label{fig:sim2}
\end{figure}

\section{Applications} \label{app:meta2}
\subsection{Gene expression signatures for breast cancer survival}
To illustrate the performance of our approach, we applied the hierarchical survival model to the Molecular Taxonomy of Breast Cancer International Consortium (METABRIC) study. The METABRIC microarray dataset is available at European Genome-Phenome Archive with the accession of EGAS00000000083. It includes cDNA microarray profiling of around 2000 breast cancer specimens processed on the Illumina HT-12 v3 platform (Illumina\_Human\_WG-v3) \citep{curtis2012genomic}. The dataset was divided into a discovery/training set of 997 samples, and a validation/test set of 995 samples \citep{cheng2013development}. The goal is to build a prognostic model for breast cancer survival, based on gene expressions and clinical features. The data $\bm{X}$ consists of of 29,477 gene expression probes and two clinical features, age at diagnosis and the number of positive lymph nodes. The meta-feature data $\bm{Z}$ consists of four ``attractor metagenes'', which are selected gene co-expression signatures that are associated with the ability of cancer cells to divide uncontrollably, to invade surrounding tissues, and, with the effort of the organism to fight cancer with a particular immune response \citep{cheng2013biomolecular}. The three universal “attractor metagenes” are genes involved in mitotic chromosomal instability (CIN), in mesenchymal transition (MES), lymphocyte-specific immune recruitment (LYM). In addition, a meta-gene whose expression is associated with good prognosis and that contains the expression values of two genes—FGD3 and SUSD3. The CIN, MES, and LYM metagenes each consist of 100 genes, but for our analysis, we only considered the 50 top-ranked genes. The data matrix $\bm{Z}$ is an indicator matrix of whether a specific expression probe corresponds to a gene in a metagene. 

Model building was based on the samples with ER positive and HER2 negative, as treatments are homogeneous in this group, and they are associated with good prognosis \citep{rivenbark2013molecular}. There were 740 samples in the discovery set and 658 samples in the validation set in the ER+ and HER2- subset after removing samples with missing values. We used 5-fold cross validation to tune the hyper-parameters $\lambda_1,\lambda_2$ in the discovery set. The test set was used to evaluate model performance. The same training/test scheme was used to fit a standard ridge regression without attractor metagene information as comparison. 

With only gene expression features in the model and no clinical features, the test C-index for the ridge-lasso hierarchical model with metagene information was 0.678 which compares favorably with the test C-index of 0.648 for the standard Cox ridge counterpart. When adding the clinical features, age at diagnosis and number of positive lymph nodes, the test C-index increased to 0.752, and 0.727 for the Cox hierarchical model, and the standard Cox ridge model, respectively (Table \ref{table1}). The metagenes``CIN'' and ``FGD3-SUSD3'' were identified by the hierarchical model as being important (had non-zero coefficients). ``CIN'', which is a breast cancer inducing metagene, had a positive coefficient, indicating genes in ``CIN'' had an overall increased risk over other genes, while the ``FGD3-SUSD'' metagene had a negative coefficient estimate, indicating FGD3 and SUSD3 had a reduced risk (Table \ref{table2}). The identified metagenes were also found by previous analysis \citep{cheng2013development}.
\begin{table}[tbh]
    \centering
    \def\arraystretch{1.5}
    \begin{tabular}{|c|c|c|c|}
        \hline
        \multicolumn{2}{|c|}{} & \textbf{Standard Ridge} & \textbf{Ridge-Lasso} \\ 
        \specialrule{.1em}{.05em}{.05em}
        \multirow{2}{*}{\textbf{Test C-index}} & Gene expressions only & 0.648 & 0.678 \\ 
        & Gene expressions + clinical features & 0.727 & 0.752 \\ 
        \hline
    \end{tabular}
    \caption{METABRIC: Test C-index between standard ridge and ridge-lasso}
    \label{table1}
\end{table}

\begin{table}[tbh]
    \centering
    \def\arraystretch{1.5}
    \begin{tabular}{|c|c c|}
        \hline
        \multirow{2}{*}{\textbf{Metagene}} & \multicolumn{2}{ c|}{\textbf{Coefficient Estimate}} \\
         & Gene expressions only & Gene expressions + clinical \\
        \specialrule{.1em}{.05em}{.05em}
        CIN & 0.0094 & 0.0080 \\
        \hline
        MES & 0.0021 & 0.0033 \\
        \hline
        LYM & 0.0011 & 0.0008 \\
        \hline
        FGD3-SUSD3 & -0.2083 & -0.1111 \\
        \hline
    \end{tabular}
    \caption{METABRIC: Coefficient estimates for metagenes}
    \label{table2}
\end{table}

\subsection{Anti-PD1 predictive biomarker for melanoma survival}
We also applied the model to a melanoma data set to predict overall survival after treating patients with anti-PD-1 immune checkpoint blockade. The programmed death 1 pathway (PD-1) is an immune-regulatory mechanism used by cancer to hide from the immune system. Antagonistic antibodies to PD-1 pathway and its ligands, programmed death ligand 1 (PD-L1), has demonstrated high clinical benefit rates and tolerability. Immune checkpoint blockades such as Nivolumab, pembrolizumab are anti-PD-1 antibodies showing improved overall survival for the treatment of advanced melanoma. However, less than 40\% of the patients respond to the treatments \citep{moreno2015anti}. Therefore, predicting treatment outcomes, identifying predictive signals are of great interest to appropriately select patients most likely to benefit from anti-PD-1 treatments. We explored transcriptomes and clinical data using our model to illustrate prediction performance and predictive signal selection.

The dataset combined 3 clinical studies in which RNA-sequencing were applied to patients treated with anti-PD1 antibodies, \cite{gide2019distinct, riaz2017tumor, hugo2016genomic}. The gene expression values are normalized toward all sample average in each study as the control, so that they are comparable to one another across features within a sample and comparable to one another across samples. There are 16010 genes in common across 3 studies and 117 subjects combined. We build predictive models in terms of overall survival, based on gene expression profile. Since the subjects are all treated with anti-PD1 antibodies, the transcriptomic features selected by the model are predictive signals for treatment efficacy or resistance. We selected meta-features from molecular signature database, hallmark gene sets \citep{liberzon2015molecular}. 13 gene sets are enriched to have false positive rates less than 0.25. An indicator matrix $\bm{Z}$ is formed to illustrate whether each of the 16010 genes belong to one of the 13 hallmark gene sets.

We performed 5-fold cross validation to tune the hyperparameters and report the validation prediction performance. We see improvement in prediction with the hallmark gene set information with a C-index of 0.663 for ridge-lasso compared to 0.637 for standard ridge when only including gene expressions; 0.754 for ridge-lasso compared to 0.739 for standard ridge when inculding gene expressions and clinical features. At the gene set level, among model selected sets (non-zero coefficient $\alpha$), 3 gene sets have absolute effect size larger than 0.01 (Table \ref{table3}). Specifically, genes in response to interferon gamma, genes that are involved in KRAS regulation were identified. A subset of the genes in the identified gene sets by our model were in concordance with the previously published anti-PD1 gene signatures \citep{riaz2017tumor, hugo2016genomic}.
\begin{table}[tbh]
    \centering
    \def\arraystretch{1.3}
    \begin{tabular}{|c|c|}
        \hline
        \textbf{Gene set} & \textbf{Coefficient estimate}  \\
        \specialrule{.1em}{.05em}{.05em}
        IFNG interferon gamma response & -0.0100* \\
        Interferon alpha response & -0.0013 \\
        IL-2\_STAT5 signaling & 0.0072 \\
        Bile acid metabolism & -0.0011 \\
        KRAS signaling down regulated & -0.0135* \\
        KRAS signaling up regulated & 0.0100* \\
        Apoptosis & 0.0004 \\
        Xenobiotic metabolism  & 0.0013 \\
        \hline
    \end{tabular}
    \caption[Anti-PD1: Coefficient estimates (nonzero) for hallmark meta-features]{
        Coefficient estimates (nonzero) for hallmark meta-features. * Gene sets with absolute value of coefficients larger than 0.01.
        }
    \label{table3}
\end{table}

\section{Discussion}
In this chapter we extended the regularized hierarchical regression model of Kawaguchi et al. (2021) to time-to-event data and to accommodate a lasso or elastic-net penalty in the second-level model. The hierarchical regularized regression model enables integration of external meta-feature information directly into the modeling process.  We showed that prediction performance improves when the external meta-feature data is informative. And the improvements are largest for smaller sample sizes, when prediction is hardest and performance improvement is most needed. Key to obtaining performance gains though is prior knowledge of external information that is potentially informative for the outcome. For example, clinicians, epidemiologists, or other substantive experts may provide insights into what type of annotations are likely to be informative.  However, the model is robust to incorporating a set of meta-features that is completely irrelevant to the outcome of interest.  In this scenario, a very small price in prediction performance is paid relative to a standard ridge model (i.e., without external information). This should encourage the user to integrate meta-features even if uncertain about their informativeness.

An underlying assumption of the proposed regularized hierarchical model is that the effects in a group determined by meta-features (e.g., genes in a pathway) are mostly in the same direction. A limitation of the method is that if the effects have opposite signs and ‘cancel each other out’ there would be little or no improvement in prediction, even if the pathway information is informative.

In addition to developing predictive signatures, the model can also be deployed in discovery applications where the main goal is to identify important features associated with the outcome rather than developing a predictive model. However, there is no standard way to perform formal inference (standard errors, p-values, confidence intervals) with high-dimensional regression models. Several approaches exist \citep{meinshausen2009p, shah2013variable} and this is an active area of research. Adding formal statistical inference would be an important future work to expand the range of use of the proposed model. 

The regularized hierarchical model is implemented in the``xrnet'' R package available from CRAN. The implementation is efficient and can be used to perform analyses with large numbers of features, meta-features, and subjects. While the models we focused on in the simulation and data applications are all “ridge-lasso”, i.e., with an $L_2$ norm penalty applied to $\bm{\beta}-\bm{Z\alpha}$, and an $L_1$ norm applied to the meta-feature coefficients $\bm{\alpha}$, the package offers the flexibility of using the Lasso, elastic net, and ridge penalties to penalize the meta-features depending on the application.  For example, if selection at the meta-feature level is desired and the meta-features are highly correlated, the elastic net penalty is a better option for $\bm{\alpha}$ regularization. Because if there is a group of variables that are highly correlated, the lasso tends to select one of them, while the elastic net enjoys grouping effect which selects all the variables in a group with estimated coefficients close to equal in magnitude \citep{zou2005regularization}. The approach does not perform feature selection on first level information as it uses a ridge penalty. In a high dimensional setting, standard regularized regression like lasso and elastic net often select relatively large numbers of features. It can then be valuable to identify groups of genes defined by meta-features that may jointly have significant predictive power for the outcome of interest. Another potential improvement of the model is to extend the range of penalty types to nonconvex penalties, such as SCAD \citep{fan2001variable}, MCP \citep{zhang2010nearly}. These penalties yield less biased effect size estimates than that of lasso and elastic net.



% Research Topic 2
\chapter{Meta-Feature Guided Regularized Regression for Survival Outcomes}
\label{cha:xtunecox}

\section{Abstract}
Regularized regression is a widely used technique for building prognostic and diagnostic models based on genomic features. Regularized regression can handle high-dimensional data common in genomic studies, and in the case of sparse regularized methods  it can also  perform feature selection. Associated with genomic features, such as gene expression, genotypes and DNA methylation, there is a great deal of functional information that, if directly incorporated into the modeling process, could yield better prediction performance. Examples of functional information are pathways and other gene sets, gene ontology annotations, and knowledge from previous studies. Functional information can often be represented as meta-features, i.e. attributes of the features rather than of the subjects. 

In this chapter, we extend the approach of \cite{zeng2021incorporating} to survival outcomes. The method can incorporate prior information in the form of meta-features to guide regularized Cox regression models. This is accomplished by letting each genomic feature to have its own `custom' penalty parameter, rather than having a single penalty parameter controlling the degree of regularization across all features. The individual penalty parameters are in turn modeled to be functions of the meta-features.  We show the benefits of the method by a simulation study and applications to two genomic studies.  

\section{Introduction}
Predicting health outcomes based on genomic profiles is an important and active area of Biomedical research. A commonly used statistical tool for developing predictive model in genomic studies is regularized regression. When the number of features is larger than the number of samples, which is usually the case for genomic data, regularization needs to be introduced so that the model complexity, relative to the amount of data available, can be controlled. Sparse regularized regression is a popular choice, as it not only shrinks the regression coefficients toward zero to make the model less complex by reducing its variance, but it also shrinks some of the coefficients with small or no effect on the outcome to exactly zero, thereby performing feature selection. The most widely used  sparse regularized regression methods are the lasso \citep{tibshirani1996regression} and the elastic net \citep{zou2005regularization}. While the lasso and elastic net share many properties, the elastic net can better deal with correlated features. Specifically,  while the lasso tends to select a single feature among a highly correlated group of features predictive of the outcome, the elastic net tends to select the entire group and make their effect-size estimates similar to each other. Ridge regression \citep{hoerl1970ridge} is another widely used regularization technique to cope with high-dimensional and collinear data. However, ridge regression does not yield sparse models, i.e. it does not perform feature selection. 

Several extensions of the lasso have been developed to exploit additional structure of the features such as groupings or natural orderings. For example, the group lasso \citep{yuan2006model} takes grouping information such as genes mapping to a pathway or probes mapping to a a gene, and shrinks the coefficients by group so that either all the features in a group are selected or no feature in the group is selected. The sparse group lasso \citep{simon2013sparse} further allows sparsity within each group. The fused lasso deals with the setting where an ordering of the features is available (e.g. along the genome), by adding an $L_1$ penalty for the differences of neighbouring coefficients. The above extend regularized  regression methods to account for attributes of the features, which can be often encoded as features of the features or meta-features. Example of meta-features common in genomic studies are functional gene sets like hallmark \citep{liberzon2015molecular}, gene ontology pathways like those in the Reactome database \citep{jassal2020reactome}, and summary statistics like p-values and regression coefficients from meta-analyses or other previous relevant work. Meta-features can be encoded as data matrices, where the samples/rows represent the features, and the columns represent the meta-features. None of the regularization approaches above can systematically utilize general meta-feature information. For example, the group lasso requires features in different groups to be mutually exclusive. The fused lasso can incorporate an ordering of the features into account, which can be captured by a single quantitative meta-feature, but it is not designed to integrate multiple quantitative meta-features. 

One way of utilizing functional meta-features that define groupings is by performing gene set enrichment analysis \citep{subramanian2005gene} after identifying features that are predictive of the outcome of interest. However, incorporating meta-features directly into the modeling process can potentially improve both prediction performance and the quality of feature selection. In chapter \ref{cha:xrnetcox}, we incorporated  meta-features in a hierarchical fashion, in an approach that can be conceptualized as regressing the outcome on the original features, 
$ \bm{Y} = \bm{X\beta} + \bm{\varepsilon} $, 
where $\bm{Y}$ is the length $n$ outcome vector, $\bm{X}$ is the $n \times p$ data matrix, $\bm{\beta}$ is the length $p$ feature coefficient vector to be estimated in the model, and then regressing the estimates $\bm{\beta}$ on the meta-features, $\bm{\beta} = \bm{Z\alpha} + \bm{\gamma}$, where $\bm{Z}$ is the $p \times q$ meta-feature matrix, $\bm{\alpha}$ is coefficients vector for meta-features. To integrate both features and meta features jointly, an optimization problem is formulated as follows: 
$$ \min_{\bm{\beta, \alpha}} \frac{1}{2n} \|\bm{Y}-\bm{X \beta} \|_2^2 + \frac{\lambda_1}{2} \|\bm{\beta} - \bm{Z \alpha} \|_2^2 + \lambda_2 \|\bm{\alpha}\|_1 $$
The feature data, $\bm{X}$, and the meta-feature data, $\bm{Z}$, are both integrated  through the $L_2$ terms. The additional $L_1$ term penalizes the meta-feature coefficients $\bm{\alpha}$ to control the model complexity and perform meta-feature selection. This approach incorporates meta-features in a linear way. In Chapter 2 it was shown that this approach can considerably improve the prediction performance of regularized Cox regression  when integrating informative meta-features. \cite{zeng2021incorporating} developed an alternative method for integrating meta-features $\bm{Z}$ in a non-linear way, for quantitative and binary outcomes. This is accomplished by letting each feature have its own `customized' penalty parameter, rather than a single global penalty parameter for all features. The customized penalty parameter approach can also improve prediction performance and feature selection. In this chapter, we extend this approach to survival outcomes. 

\section{Methods}
\subsection{Setup and notation}
Starting with the time-to-event model setup, we let the outcome be $(\bm{y, \delta})$ where $\bm{y}=(y_1,y_2,\dots,y_n)$ denotes the vector of observed times, and $\bm{\delta}=(\delta_1,\delta_2,\dots,\delta_n)$ denotes the vector of censoring status for the $n$ subjects. For subject $i$, $i=1,2,\dots,n$, $\delta_i = 1$  if the event of interest (e.g., death) took place within the followup period and $y_i$ is time of the event; if $\delta_i=0$, the event did not happen within the followup period, and $y_i$ is the censoring time. We assume there are $p$ genomic features (e.g. expression or methylation levels, genotypes), measured on each of the $n$ subjects. The  $n\times p$ data matrix $\bm{X}$ stores the feature values, i.e., $\bm{x}_i = (x_{i1},x_{i2},\dots,x_{ip})$ is a vector of feature values for subject $i$. Associated with the features, we assume there are $q$ meta-features. A $p\times q$ matrix $\bm{Z}$ stores the meta-feature values for the $p$ original features, i.e., $\bm{z}_j = (z_{j1},z_{j2},\dots,z_{jq})$ for  $j=1,2,\dots,p$ is a vector of meta-feature values for feature $j$. The most common  regression method for time-to-event data is Cox's proportional hazards model \citep{cox1972regression}. It assumes proportional hazard functions with respect to feature values at the same time point, which allows model fitting without explicitly knowing the form of the baseline hazard function, and  depending only on the order in which events occur, but not on the exact time of occurrence. We let $t_1 \le t_2 \le \dots \le t_l \le \dots \le  t_m$ be the event times arranged in increasing order and $D_l=\{i:\delta_i=1,y_i=t_l\}$ be the set of subjects that experienced the event at time $t_l$. Letting $\bm{\beta}$ be a length $p$ vector of regression coefficients, Breslow's adjustment to the Cox partial likelihood \citep{breslow1972contribution} takes the form: 
\begin{displaymath}
L(\bm{\beta}) = \prod_{l=1}^{m} \frac{e^{\sum_{i\in D_l}\bm{x}_i^T\bm{\beta}}}{(\sum_{i\in R_l} e^{\bm{x}_i^T\bm{\beta}})^{d_l}}
\end{displaymath}
where $R_l=\{i: y_i\geq t_l\}$ is the risk set at event time $t_l$, i.e., the set of all subjects who have not experienced the event and are uncensored just prior to time $t_l$, and $d_l=|D_l|$ is the number of events at time $t_l$.  Breslow's partial likelihood handles potential ties at each event time (i.e., more than one subject experiencing the event at the same time). When there are no ties, $L(\bm{\beta})$ automatically reduces to Cox's partial likelihood. We can see that neither the hazard functions nor the actual times are involved in the partial likelihood, only the order of event times matters. 

To control the model complexity we consider an elastic net regularized Cox model. Denoting the log partial likelihood by $\ell(\bm{\beta})$, the regularized Cox regression model is the solution to the following optimization problem:
\begin{equation} \label{eq1}
    \min_{\bm{\beta}\in \mathbb{R}^p} \left\{-\ell(\bm{\beta}) + \lambda\left[\frac{1}{2}(1-c)\|\bm{\beta}\|_2^2 + c\|\bm{\beta}\|_1\right]\right\}.
\end{equation}
The regularization function includes the lasso ($c=1$), elastic net ($0 < c <1)$, and ridge ($c=0$) penalties as particular cases. When $ 0<c \leq 1$, the penalty is sparse inducing, i.e.,  shrink some coefficients to exactly zero, producing more interpretable models. In \eqref{eq1} the global penalty parameter $\lambda$ is the same  for all the features, which implicitly assumes all features are a priori equally important.  To incorporate meta-features which might be informative about the importance of the features, we give each feature its own unique penalty parameter $\lambda_j$, which in turn we model to be a function of the meta-features. Specifically,  we assume $\lambda_j=e^{\bm{z_j}^T \bm{\alpha}}$ where $\bm{\alpha}$ is a weight vector of length $q$. The model is now given by the following optimization problem: 
\begin{equation} \label{eq2}
\begin{aligned}
    &\min_{\bm{\beta}\in \mathbb{R}^p} \left\{-\ell(\bm{\beta}) + \sum_{j=1}^p \lambda_j\left[\frac{1}{2}(1-c)\beta_j^2 + c|\beta_j|\right]\right\}, \\
    &\lambda_j = e^{\bm{z_j}^T \bm{\alpha}}.
\end{aligned}
\end{equation}

\subsection{Model fitting}
The standard elastic net Cox proportional hazards model in equation \eqref{eq1}, can be fitted by pathwise coordinate descent \citep{simon2011regularization}. Since the  penalty parameter $\lambda$ is a global hyper-parameter, the algorithm typically constructs a grid of $\lambda$ values from which the best is selected by cross-validation. By contrast, the proposed model in equation \eqref{eq2}, has $p$ $\lambda$'s defined by the weights as $\bm{\alpha}$, $\bm{\lambda} = (\lambda_1,\lambda_2,\dots,\lambda_p) = e^{\bm{Z\alpha}}$, so it is not feasible to tune them using traditional approaches like cross-validation. Instead, we estimate the weights $\bm{\alpha}$ to get the  $\bm{\lambda}$ directly based on the data using an empirical-Bayes approach. With known $\bm{\lambda}$, the model in \eqref{eq2} can be fit via coordinate descent.

\subsubsection{Empirical Bayes estimation of the hyperparameters} \label{laplace}
To estimate $\bm{\alpha}$, we rely on an alternative and natural Bayesian/random effects formulation of the model in \eqref{eq2}. We then estimate the hyper-parameters by maximizing the marginal likelihood obtained by integrating out the random effects $\bm{\beta}$. Based on the Bayesian elastic net \citep{li2010bayesian}, the model given by equation \eqref{eq2} can be re-formulated as the following hierarchical model:
\begin{align}
    &f(\bm{Y}|\bm{\beta}; \bm{X}) = \prod_{l=1}^{m} \frac{e^{\sum_{i\in D_l}\bm{x}_i^T\bm{\beta}}}{(\sum_{i\in R_l} e^{\bm{x}_i^T\bm{\beta}})^{d_l}} \label{eq3}, \\
    &\pi(\beta_j; \bm{\alpha}) \propto exp\left\{ -\lambda_j\left[\frac{1}{2}(1-c)\beta_j^2 + c|\beta_j|\right] \right\}. \label{eq4}
\end{align}
With the likelihood \eqref{eq3} and prior distribution \eqref{eq4}, we construct the joint distribution of $\bm{Y}$ and $\bm{\beta}$, and integrate out $\bm{\beta}$, so to get the marginal likelihood of $\bm{Y}$:
\begin{align*}
f(\bm{Y};\bm{\alpha}) &= \int_{\bm{\beta}\in\mathbb{R}^p} f(\bm{Y}, \bm{\beta};\bm{\alpha}) d\bm{\beta} \\
&= \int_{\bm{\beta}\in\mathbb{R}^p} f(\bm{Y}|\bm{\beta};\bm{X}) \pi(\bm{\beta};\bm{\alpha})d\bm{\beta} \\
&= \int_{\bm{\beta}\in\mathbb{R}^p} \prod_{l=1}^{m} \frac{e^{\sum_{i\in D_l}\bm{x}_i^T\bm{\beta}}}{(\sum_{i\in R_l} e^{\bm{x}_i^T\bm{\beta}})^{d_l}} \prod_{j=1}^{p}exp\left\{ -\lambda_j\left[\frac{1}{2}(1-c)\beta_j^2 + c|\beta_j|\right] \right\}  d\bm{\beta}
\end{align*} 
This integral does not have a closed form expression because the elastic net prior is not conjugate for the likelihood. We propose to approximate the integral by first approximating the elastic net prior with a normal prior, and then  applying the Laplace approximation to get a tractable final approximation. To approximate the elastic net prior, we use a normal prior with the same variance as the lasso component. The lasso component of the prior corresponds to a double exponential distribution, $DE(\lambda_jc)$, with variance $\frac{2}{(\lambda_jc)^2}$. The normal approximation of this double exponential distribution is then $N(0, \frac{2}{(\lambda_jc)^2})$. Hence, the elastic net prior can be approximated as follow,
\begin{equation} \label{eq5}
\begin{aligned}
    \pi(\beta_j; \bm{\alpha}) &\propto exp\left\{ -\lambda_j\left[\frac{1}{2}(1-c)\beta_j^2 + c|\beta_j|\right] \right\} \\
    &\approx exp\left\{ -\frac{1}{2}\lambda_j(1-c)\beta_j^2 + \frac{\beta_j^2}{2(\frac{2}{(\lambda_jc})^2} \right\} \\
    &= N \left( 0, \frac{2}{2\lambda_j(1-c)+c^2\lambda_j^2} \right). 
\end{aligned}
\end{equation}
\begin{figure}[tbh]
  \centering
  \includegraphics[width=\textwidth]{norm_appr}
  \caption{Normal approximation of the elastic net prior}
  \label{fig:norm_appr}
\end{figure}
Figure \ref{fig:norm_appr} shows the normal prior approximation compares to the elastic net prior. With the prior distribution approximation, the joint log-distribution, $\ln f(\bm{Y}, \bm{\beta};\bm{\alpha})$, then takes the form of a ridge regularized Cox regression with individual penalty parameters,
\begin{equation} \label{eq6}
\begin{aligned}
    &\ln f(\bm{Y}, \bm{\beta};\bm{\alpha}) = \sum_{l=1}^{m}\left[\sum_{i\in D_l}\bm{x}_i^T\bm{\beta}-d_l\ln(\sum_{i\in R_l} e^{\bm{x}_i^T\bm{\beta}})\right] - \sum_{j=1}^{p} \frac{1}{2}v_j\beta_j^2 + \text{const}, \\
    &v_j = \frac{2\lambda_j(1-c)+c^2\lambda_j^2}{2}.
\end{aligned}
\end{equation}
Applying the Laplace approximation \citep{laplace1986memoir} to \eqref{eq6}, the integral with respect to $p$-dimensional vector $\bm{\beta}$ has a closed form solution. Consider a Taylor series of $\ln f(\bm{Y}, \bm{\beta};\bm{\alpha})$ at the stationary point $\tilde{\bm{\beta}}$, where $\nabla \ln f(\bm{Y}, \tilde{\bm{\beta}};\bm{\alpha})=0$,
$$ \ln f(\bm{Y}, \bm{\beta};\bm{\alpha}) \approx \ln f(\bm{Y}, \tilde{\bm{\beta}};\bm{\alpha}) - \frac{1}{2}(\bm{\beta}-\tilde{\bm{\beta}})^T\bm{H}(\bm{\beta}-\tilde{\bm{\beta}}). $$
$\tilde{\bm{\beta}}$ is the solution of a ridge regularized Cox regression \eqref{eq6}, which can be computed, for fixed $\bm{\alpha}$,  using for example the  \emph{glmnet} R package \citep{simon2011regularization}. Here $\bm{H}$ is the Hessian matrix,
\begin{align*}
    \bm{H} &= - \nabla\nabla|_{\bm{\beta}} \ln f(\bm{Y}, \bm{\beta};\bm{\alpha})|_{\bm{\beta}=\tilde{\bm{\beta}}} \\
    & \approx \bm{X}^T\bm{W}\bm{X} + \bm{V}
\end{align*}
where $\bm{V} = \text{diag}[\bm{v}]=\text{diag}[v_1,\dots,v_p]$, $\bm{W}$ is a diagonal matrix with elements 
$$ \bm{W}_{ii} = \sum_{l\in C_i}\frac{d_le^{\bm{x}_i^T\bm{\beta}}}{\sum_{k\in R_l}e^{\bm{x}_k^T\bm{\beta}}} - \sum_{l\in C_i}\frac{d_l(e^{\bm{x}_i^T\bm{\beta}})^2}{(\sum_{k\in R_l}e^{\bm{x}_k^T\bm{\beta}})^2}. $$ 
The Hessian is in turn approximated to avoid computing the full matrix $W$, which is costly. Here we only use diagonal elements to speed up the computation without loss of accuracy. For greater details, refer to \cite{simon2011regularization}.
We see now that $f(\bm{Y}, \bm{\beta};\bm{\alpha})$'s Taylor approximation has a multivariate normal form with mean $\tilde{\bm{\beta}}$, and variance $\bm{H}^{-1}$. Integrating out $\bm{\beta}$ yields the normalizing constant:
\begin{equation} \label{eq7}
\begin{aligned}
    -\ln{f(\bm{Y};\bm{\alpha})} &\approx -\ln f(\bm{Y}|\tilde{\bm{\beta}};\bm{X}) - \ln \pi(\tilde{\bm{\beta}};\bm{\alpha}) - \frac{p}{2}\ln{2\pi} + \frac{1}{2}\ln{|\bm{H}|} \\
    &= -\ln{|\bm{V}|} + \tilde{\bm{\beta}}^T\bm{V}\tilde{\bm{\beta}} + \ln{|\bm{H}|} + \text{const}
\end{aligned}
\end{equation}
The approximate negative log marginal likelihood in equation \eqref{eq7} is the objective function we  minimize to estimate $\bm{\alpha}$.

\subsubsection{Marginal likelihood function optimization} \label{DCA}
Although the marginal likelihood function given by equation \eqref{eq7} is nonconvex, it can be decomposed as the difference of two convex functions: $g(\bm{\alpha}):=-\ln{|\bm{V}|} + \tilde{\bm{\beta}}^T\bm{V}\tilde{\bm{\beta}}$, and  $-h(\bm{\alpha}):=-\ln{|\bm{H}|}$. This decomposition makes it possible to use optimization algorithms specifically designed for difference of convex functions (DCA) \citep{le2015dc}. The main idea of DCA is to approximate the nonconvex objective function by a sequence of convex ones. At each iteration  we approximate the concave part ($h(\bm{\alpha}$)) by its affine majorization, i.e., the supporting hyperplane obtained by calculating its gradient, or subgradients if not differentiable, and then minimize the resulting convex approximation. The resulting algorithm can also be thought of as a majorization-minimization algorithm \citep{hunter2004tutorial}. The affine approximation of the concave part is the majorization step, which forms a surface lying above the objective function, and is tangent to it, i.e, at the current estimation of the target parameter, the majorization equals to the objective function. This ensures the majorization is a tight upper bound for the objective. Minimizing the convex upper bound is the minimization step. The DCA algorithm for the marginal likelihood $-\ln{f(\bm{Y};\bm{\alpha})}$:
\begin{enumerate}
    \item Initialize $\bm{\alpha}$ with $\tilde{\bm{\alpha}} \in \mathbb{R}^q$.
    \item Majorization: 
    \begin{itemize}
        \item calculate the gradient at current estimation $\tilde{\bm{\alpha}}$,
    $$\bm{\theta}= \nabla_{\bm{v}} \ln{|\bm{H}|} = \text{diag}[\bm{H}^{-1}]$$ 
        \item form the convex upperbound,
        \begin{align*}
        u(\bm{\alpha})&=g(\bm{\alpha})+ h(\tilde{\bm{\alpha}}) + \bm{\theta}^T(\bm{v}-\tilde{\bm{v}}) \\
        &=-\ln{|\bm{V}|} + \tilde{\bm{\beta}}^T\bm{V}\tilde{\bm{\beta}}+\bm{\theta}^T\bm{v}+\text{const}
        \end{align*}
    \end{itemize}
    \item Minimization: $\hat{\bm{\alpha}}=\underset{\bm{\alpha}}{\operatorname{\argmin}} \left\{u(\bm{\alpha})\right\}$.
    \item Set $\tilde{\bm{\alpha}} = \hat{\bm{\alpha}}$.
    \item Repeat step 2-4 until convergence of $\hat{\bm{\alpha}}$.
\end{enumerate}
The minimization of $u(\bm{\alpha})$ can be performed with a standard first order method like gradient descent, or a second order method like Newton-Raphson. The gradient and Hessian of $u(\bm{\alpha})$ are:
\begin{align*}
    &\nabla_{\bm{\alpha}} u(\bm{\alpha}) = \bm{Z}^T\left[(-\frac{1}{\bm{v}}+\tilde{\bm{\beta}}^2+\bm{\theta})((1-c)\bm{\lambda}+c^2\bm{\lambda}^2)\right],\\
    &\nabla\nabla_{\bm{\alpha}} u(\bm{\alpha}) = \bm{Z}^T \text{diag}\left[\frac{\bm{\lambda}^2}{\bm{v}^2}(1-c+c^2\bm{\lambda})^2+(-\frac{1}{\bm{v}}+\tilde{\bm{\beta}}^2+\bm{\theta})\bm{\lambda}(1-c+2c^2\bm{\lambda})\right]\bm{Z}.
\end{align*}

\subsection{Summary} \label{cha3_sum}
We incorporate the meta-features into feature-specific penalty parameters modeled as a log-linear functions of the meta-features. We then use a Bayesian interpretation of regularized regression to obtain the marginal likelihood function for the meta-feature weights $\bm{\alpha}$. The $\bm{\alpha}$ parameter are estimated by maximum marginal likelihood. The resulting nonconvex objective function can be decomposed into a difference of two convex functions, which can be optimized with a difference of convex functions algorithm. The estimated $\bm{\alpha}$ can be plugged in  to compute the penalty parameters. The complete model fitting procedure is given by:
\begin{enumerate}
    \item Initialize $\bm{\alpha}$ with $\tilde{\bm{\alpha}}$.
    \item Repeat, until convergence of $\hat{\bm{\alpha}}$.
    \begin{enumerate}
        \item Laplace approximation of marginal likelihood with known $\tilde{\bm{\alpha}}$, section \ref{laplace},
        \begin{itemize}
            \item Approximate the elastic net prior with a normal prior, equation \eqref{eq6},
            \item Calculate $\tilde{\bm{\beta}}$ and $\bm{H}$.
        \end{itemize}
        \item Optimize Laplace approximation of marginal likelihood, equation \eqref{eq7}, get solution $\hat{\bm{\alpha}}$, with DCA described in section \ref{DCA}.
        \item Set $\tilde{\bm{\alpha}} = \hat{\bm{\alpha}}$.
    \end{enumerate}
    \item Calculate customized penalty vector $\bm{\lambda}=e^{\bm{Z}\hat{\bm{\alpha}}}$.
    \item Fit regularized Cox regression, equation \eqref{eq2}, with $\bm{\lambda}$.
\end{enumerate}

\subsection{Inclusion of unpenalized features}
We have introduced the computational algorithm for meta-feature guided regularized regression \eqref{eq2}, in which all the features are regularized, and have meta-features associated with them. However, more often than not, there are other types of features providing substantial predictive power to the outcome, in addition to genomics. For example, demographics like age at diagnosis, gender; clinical features such as symptoms, imaging and lab results. These features usually do not have meta-feature information and it is desirable to include them in the model without penalization. To extend our model to include unpenalized features, we denote unpenalized feature matrix $\bm{X}'_{n\times p'}$, i.e., there are $p'$ unpenalized features; $\bm{\beta}'\in \mathbb{R}^{p'}$ is the coefficient vector for them. Then \eqref{eq2} can be rewritten as
\begin{equation} \label{eq8}
\begin{aligned}
    &\min_{\bm{\beta}'\in \mathbb{R}^{p'}, \bm{\beta}\in \mathbb{R}^p} \left\{-\ell(\bm{\beta}', \bm{\beta}) + \sum_{j=1}^p \lambda_j\left[\frac{1}{2}(1-c)\beta_j^2 + c|\beta_j|\right]\right\}, \\
    &\lambda_j = e^{\bm{z_j}^T \bm{\alpha}}.
\end{aligned}
\end{equation}
The only difference between \eqref{eq8} and \eqref{eq2} is the addition of unpenalized features $\bm{X}'$ and their coefficients $\bm{\beta}'$ in log partial likelihood $\ell(\bm{\beta}', \bm{\beta})$. If we walk through the algorithm described in section \ref{cha3_sum}, the procedures will follow. Except that the alternating optimization of $\bm{\beta}$, $\bm{\alpha}$ now takes place with $(\bm{\beta}', \bm{\beta})$, $\bm{\alpha}$, and $\bm{\beta}'$ is estimated using regularized regression with their $\lambda$ being 0.

\section{Simulations}
\subsection{Simulation methods}
In this section, we perform a series of simulation experiments to evaluate the performance of the proposed model in terms of prediction and feature selection performance compared to standard regularized Cox regression. We first generate meta-feature data $\bm{Z}$ by sampling from independent Bernoulli variables, with probability 0.1. This is to mimic biological pathway/functional gene set meta-features. Each pathway contains a group of genes, and 1 indicates the gene belongs to the pathway and 0 if the gene does not. Meta-feature weights $\bm{\alpha}$ were fixed from an equally spaced grid of values ranging from -1 to 1. We then generate $\bm{\beta}$ from a normal prior distribution with mean 0, and variance computed from $\bm{\alpha, Z}$, based on equation \eqref{eq5}. Because we want the underlying model to be sparse, we only keep the top 20\% of the $\bm{\beta}$ elements with largest absolute values, and set the remaining to be 0. To model uninformative meta-features we also consider scenarios where we randomly flip some of the rows of $\bm{Z}$ from 0 to 1 or vice versa. The proportion of modified rows, tracks the overall informativeness of meta-features, e.g., 10\% of the rows modified corresponds to high informativeness while 90\% of the rows modified corresponds to low informativeness. The rows of the data matrix $\bm{X}$ are sampled from a multivariate normal distribution with autoregressive correlation structure, $\bm{\Sigma}_{ij} = \rho^{|i-j|}$, with $\rho=0.5$. Survival times are generated using the inverse probability integral transform:   
\begin{displaymath}
t = H_0^{-1}\left(-\ln(U)e^{-\bm{\beta}^T\bm{x}}\right)
\end{displaymath}
where $U\sim \text{uniform}[0,1]$, $H_0(t) = (t/5)^8$ is a baseline cumulative hazard function with Weibull distribution. Censoring times are sampled from an exponential distribution, $c\sim \text{exp}(0.1)$. The parameter values for the Weibull and exponential distributions are set to generate survival times in the 0 to 20 range, and a fixed ratio (3/2) of events to censoring. We then add normal noise to survival times to control the overall underlying predictive ability of the features for the outcome. Specifically, we set the concordance measure \citep{harrell1982evaluating} to 0.8, by controlling the standard deviation of the added noise. The survival outcome is set to be the minimum of the generated survival and censoring time, $y=\text{min}(t,c)$.

For each simulation replicate, we generated training data as described above. We then fit a standard elastic net regularized Cox regression without external meta-features $\bm{Z}$, and also fit our proposed model with meta-features. Elastic net regression is tuned with 5-fold cross validation, while the proposed model does not require penalty parameter tuning as it is estimated as part of the model fitting procedure. We then compare the prediction performance between the two models on an independent test set of size 1000 generated following the same procedure described for the training data. We used the C-index as a performance metric. For each simulation scenario we used 100 replicates. 

We run a series of experiments varying one key parameter while keeping the others fixed. The base case parameters are sample size $n=100$, feature size $p=200$, meta-feature size $q=10$, meta-feature $\bm{Z}$ informativeness: 5\% of features (rows of $\bm{Z}$) modified to have incorrect values. Four experiments are conducted by varying one parameter at a time.
\begin{enumerate}
    \item Meta-feature informativeness level from high to low, proportion of rows of $\bm{Z}$ modified 5\%, 15\%, 30\%.
    \item Feature size, $p=200, 600, 1000$.
    \item Sample size, $n=100, 200, 300$.
    \item Meta-feature size, $q=10, 20, 30$.
\end{enumerate}
In experiment 1, we also examined the feature selection performance by both models, to evaluate how informativeness of meta-features influences model interpretation.

\subsection{Simulation results}
Figure \ref{fig:sim21} shows the results of the 4 simulation experiments. The horizontal dashed line in each panel represents the population/theoretical C-index, achievable with infinite samples of training data. It is provided as a reference for each parameter setting. The performance of the meta-feature guided elastic net Cox model (denoted as `meta' in the figure) is compared to that of a standard elastic net Cox model. 

In experiment 1, there are consistent improvements in prediction performance as long as the meta-features are informative. The higher the informativeness of meta-features, the larger the gains in prediction performance over the standard elastic net model.

Experiment 2 illustrates the model performance with respect to the number of features $p$. Larger number of features relative to the sample size makes it harder for both models to predict well, as both models' test C-index are substantially below the theoretical C-index of 0.8, when $p=600$ or $p=1,000$. However, the meta-feature model consistently outperforms the standard elastic net model.

Experiment 3 evaluates a similar situation as experiment 2, but instead of varying the number of features, it varies the sample size $n$ while keeping $p=200$. As the sample size gets larger, i.e, the ratio of number of features to sample size  becomes smaller, both models perform better, and the meta-feature guided elastic net consistently outperforms the standard elastic net. Furthermore, a larger sample size yields more stable prediction metrics (smaller variance of test C-index). 

As the size of meta-features increases (experiment 4), the prediction performance improvement over the standard elastic net becomes smaller. This indicates the meta-feature guided model's inability to handle a large number of meta-features.  

In terms feature selection performance, we define accurate selection as follow: features with non-zero simulated coefficients are estimated with non-zero values and features with zero simulated coefficients are estimated as zero. In Figure \ref{fig:sim22}, we compared the feature selection accuracy of the meta-feature guided elastic net model, standard elastic net model with the $\lambda$ value that gives maximum cross-validated C-index (denoted as `enet.min'), and with the $\lambda$ value that gives the most regularized model such that the cross-validated C-index is within one standard error of the maximum (denoted as `enet.1se'), in experiment 1, where the informativeness of meta-features varies. The `1se' elastic net model is sparser (fewer nonzero coefficients) than the `min' elastic net model. The proposed meta-feature guided elastic model again outperforms both standard elastic net models in feature selection accuracy. Moreover, the selection is more stable with meta-features compared to either of the standard elastic net models. Note that the selection accuracy of the `min' elastic net model is highly unstable, since it yields less sparse models than the `1se' elastic net, and the meta-feature guided model.
\begin{figure}
    \includegraphics[width=\textwidth]{sim21}
    \caption[Simulation results (meta guided): prediction performance] {Simulation result: prediction performance}
    \label{fig:sim21}
\end{figure} 

\begin{figure}
    \centering
    \includegraphics[scale=0.7]{sim22}
    \caption[Simulation results (meta guided): feature selection]{Simulation results: feature selection. `enet.min' is the standard elastic net model with the $\lambda$ value that gives maximum cross-validated C-index; `enet.1se' is the elastic net model with the $\lambda$ value that gives the most regularized model such that the cross-validated C-index is within one standard error of the maximum.}
    \label{fig:sim22}
\end{figure}

\section{Applications}
\subsection{Gene expression signatures for breast cancer survival}
To illustrate the performance of our approach in real data, we applied the meta-feature guided regularized regression to the Molecular Taxonomy of Breast Cancer International Consortium (METABRIC) study. The data includes cDNA microarray profiling for close to 2,000 breast cancer specimens processed on the Illumina HT-12 v3 platform (Illumina\_Human\_WG-v3) \citep{curtis2012genomic}. The data was divided into a discovery/training set of 997 samples, and a validation/test set of 995 samples. The goal is to build a prognostic model for breast cancer survival, based on gene expressions and clinical features. The feature data $\bm{X}$ consists of of 29,477 gene expression probes and two clinical features, age at diagnosis and the number of positive lymph nodes. The meta-feature data $\bm{Z}$ consists of four ``attractor metagenes'', which are selected gene co-expression signatures  associated with the ability of cancer cells to divide uncontrollably, to invade surrounding tissues, and, with the effort of the organism to fight cancer with a particular immune response \citep{cheng2013biomolecular}. Three metagenes are universal “attractor metagenes” and consist of  genes involved in mitotic chromosomal instability (CIN), in mesenchymal transition (MES), and lymphocyte-specific immune recruitment (LYM) respectively. A fourth metagene is associated with good prognosis and contains two genes: FGD3 and SUSD3. The CIN, MES, and LYM metagenes each consist of 100 genes, but for our analysis, we considered only the 50 top-ranked within each metagene. The data matrix $\bm{Z}$ is an indicator matrix for whether a specific expression probe corresponds to a gene in a metagene. 

Model building was based on the samples with ER positive and HER2 negative, as treatments are more homogeneous in this group, and they are associated with good prognosis \citep{rivenbark2013molecular}. There were 740 samples in the discovery set and 658 samples in the validation set in the ER+ and HER2- subset after removing samples with missing values. We applied meta-feature guided elastic net regression to train the model. The test set was used to evaluate model performance. The same training/test scheme was used to fit a standard elastic net regression without attractor metagene information as comparison. 

With only gene expression features in the model and no clinical features, the test C-index for the metagene guided elastic net model is 0.637 compared to the test C-index of 0.663 for the standard Cox elastic net counterpart. When adding the clinical features, age at diagnosis and number of positive lymph nodes, the test C-index increased to 0.715, and 0.728 for the metagene guided elastic net model, and the standard Cox elastic net model, respectively (Table \ref{table1}). Our metagene guided elastic net model does not perform as well as the standard elastic net model, in both situations, with or without clinical features. This contrasts with the experience with `xrnet', which improved performance in the METABRIC data. To understand the difference, we looked at the metagene weights estimated from the empirical Bayes method. Intercept represents the penalty applied to probes that do not map to any of the metagenes. The metagene weights indicates the differential penalty amount applied to the genes in the metagene; a positive (negative) weight will have the genes in the metagene penalized more (less) strongly, indicating they are less (more) important for predicting the mortality. The metagenes `CIN' and `FGD3-SUSD3' are more important than `MES' and`LYM', conforming to the findings with `xrnet' (chapter \ref{cha:xrnetcox}, section \ref{app:meta2}). Also, the number of features selected by metagene guided model is much less than that of standard elastic net, 9 vs 30 for gene expression only, and 9 vs 49 for gene expression plus clinical features, respectively. These findings suggests that although the methods agree in identifying which  metagenes are more important, non-linear integration of meta-features does not fit well the specific data and that the underlying relationship between gene expressions and metagenes in METABRIC is better captured by the linear model implicit in `xrnet'.  
\begin{table}[tbh]
    \centering
    \def\arraystretch{1.5}
    \begin{tabular}{|c|c|c|c|}
        \hline
        \multicolumn{2}{|c|}{} & \textbf{Elastic net} & \textbf{Metagene elastic net} \\ 
        \specialrule{.1em}{.05em}{.05em}
        \multirow{2}{*}{\textbf{C-index}} & Gene expressions only & 0.663 & 0.637 \\ 
        & Gene expressions + clinical & 0.728 & 0.715 \\ 
        \hline
    \end{tabular}
    \caption{METABRIC: Test C-index between standard elastic net and metagene guided elastic net}
    \label{table21}
\end{table}

\begin{table}[tbh]
    \centering
    \def\arraystretch{1.5}
    \begin{tabular}{|c|c c|}
        \hline
        \multirow{2}{*}{\textbf{Metagene}} & \multicolumn{2}{ c|}{\textbf{Metagene weights $\alpha$}} \\
         & Gene expressions only & Gene expressions + clinical \\
        \specialrule{.1em}{.05em}{.05em}
        Intercept & 4.7355 & 4.7582 \\
        \hline
        CIN & -1.3677 & -1.4057 \\
        \hline
        MES & 0.5520 & 0.7581 \\
        \hline
        LYM & 0.1074 & 0.2953 \\
        \hline
        FGD3-SUSD3 & -2.0187 & -1.8620 \\
        \hline
    \end{tabular}
    \caption{METABRIC: Metagene weights $\alpha$}
    \label{table22}
\end{table}

\subsection{Anti-PD1 predictive biomarker for melanoma survival}
We also applied the proposed meta-feature model to a melanoma data set to predict overall survival in patients treated with PD-1 immune checkpoint blockade. The programmed death 1 pathway (PD-1) is an immune-regulatory mechanism used by cancer to hide from the immune system. Antagonistic antibodies to PD-1 pathway and its ligands, programmed death ligand 1 (PD-L1), demonstrate  clinical benefits and tolerability. Immune checkpoint blockades such as Nivolumab, pembrolizumab are anti-PD-1 antibodies showing improved overall survival for the treatment of advanced melanoma. However, less than 40\% of the patients respond to them \citep{moreno2015anti}. Therefore,  identifying predictive signals of treatment outcomes is of great interest to select patients most likely to benefit from anti-PD-1 treatment. We explored transcriptomes and clinical data using our model to illustrate prediction performance and predictive signal selection.

The dataset combined 3 clinical studies in which RNA-sequencing profiles of patients treated with anti-PD1 antibodies was obtained: \cite{gide2019distinct, riaz2017tumor, hugo2016genomic}. The gene expression levels were normalized toward all sample average in each study as the control, so that they are comparable to one another across features within a sample and comparable to one another across samples. There are 16,010 genes in common across the 3 studies, and a combined total of 117 subjects. The clinical variables being considered are age, gender, and tumor response. We build predictive models for overall survival based on transcriptomics and clinical variables. Since the subjects are all treated with anti-PD1 antibodies, the transcriptomic features selected by the model are predictive signals for treatment efficacy or resistance. We selected hallmark gene sets as meta-features from the molecular signature database  \citep{liberzon2015molecular}. A total of 13 gene sets were enriched \citep{subramanian2005gene} with false positive rates less than 0.25 (Table \ref{table3.1}). The meta-feature matrix is formed as an indicator of whether each of the 16,010 genes belong to one of the 13 hallmark gene sets ($\bm{Z}$). 

We compared prediction and feature selection performance between the meta-feature guided elastic net model and the standard elastic net model. The data is split into training ($75\%$) and test set ($25\%$). Standard elastic net was trained using 5-fold cross validation, while the meta-feature model was trained with estimated hyperparameters. The test concordance index is C = 0.7340 for the elastic net, and C = 0.7609 for our meta-feature guided model. As for feature selection, the meta-feature model, which selects 4 genes (GPAA1, COX6C, VPS28, PLCB4), is sparser, compared to 11 genes selected in the elastic net model. 
\begin{table}[tbh]
    \centering
    \def\arraystretch{1.3}
    \begin{tabular}{|c|c|}
    \hline
     \bf Hallmark meta-feature & \bf Estimated $\bm{\alpha}$ \\
     \specialrule{.1em}{.05em}{.05em}
     Interferon gamma response & -0.0102  \\ \hline
     Allograft rejection & 0.1570  \\ \hline
     Interferon alpha response & -0.0314 \\ \hline
     IL6 JAK STAT3 signaling & 0.1131 \\ \hline
     Inflammatory response & 0.0744 \\ \hline
     Complement & 0.1577 \\ \hline
     TNFA signaling via NFKB & 0.1180 \\ \hline
     IL2 STAT5 signaling & 0.1613 \\ \hline
     Bile acid metabolism & 0.0876 \\ \hline
     Kras signaling down & 0.2338 \\ \hline
     Xenobiotic metabolism & 0.2598 \\ \hline
     Apoptosis & 0.2557 \\ \hline
     Kras signaling up & 0.2737 \\ \hline
    \end{tabular}
    \caption[Anti-PD1: List of hallmark meta-features and their respective estimated weight]{Anti-PD1: List of hallmark meta-features and their respective estimated weight $\bm{\alpha}$.}
    \label{table3.1}
\end{table}

\section{Discussion}
In this chapter, we extended to survival outcomes the customized regularization model guided by genomic meta-features of \cite{zeng2021incorporating}. This model has individualized penalty parameters for each of the features instead of the single global penalty parameter traditional in commonly used regularized regression methods. Differential penalization allows for the importance of each feature in predicting the outcome of interest to be determined from the data by their underlying characteristics/meta-features. With  informative meta-features, the more important features will be penalized less, resulting in a lower  likelihood of being shrunk to exactly 0, and for the less important features to be more strongly penalized, resulting in a higher likelihood of being excluded from the model. In our simulation experiments and the anti-PD1 immunotherapy predictive modeling application, the customized regularization model showed benefits in prediction performance, and the selected model tended to be sparser (fewer features selected). Of note is that to derive prediction performance gains, the external meta-feature data needs to be  informative and low-dimensional. Therefore,   substantive pior knowledge to carefully select a potentially relevant and limited meta-features set is critical. 

We have developed two different methods to systematically integrate meta-feature information into the modeling process with survival outcomes and high-dimensional data. The approach in this chapter integrates meta-features in a non-linear way. It allows the meta-features to dictate the importance of each feature by defining the penalty parameters as a non-linear function of the meta-features. By contrast, the `xrnet' regularized hierarchical model in chapter \ref{cha:xrnetcox} integrates meta-features linearly. The fundamental difference between the two methods is that `xrnet' models meta-features through the mean of $\bm{\beta}$, while the proposed model in this chapter models meta-features through the variance of $\bm{\beta}$. Specifically, `xrnet' models $\bm{\beta} \sim N(\bm{Z\alpha}, \tau\bm{I})$, and the customized regularization approcah models $\bm{\beta}$ as having an approximate prior $N(0, \frac{2}{2\lambda_j(1-c)+c^2\lambda_j^2})$. In most scenarios, a user would not know which model to favor in advance. Depending on the application at hand, one would want to consider both approaches, selecting the best based on the performance on test data. 

Our proposed model applied empirical Bayes to estimate penalty parameters for each feature, which uses the Laplace approximation to obtain the marginal likelihood. However, the Laplace approximation works well when sample size is relatively large. We see in simulation experiment 2, that while our model shows prediction benefits in all feature set sizes, there is a decreasing gain trend over standard elastic net Cox regression, which maybe caused by the drop in the quality of the Laplace approximation when the sample size is  small relative to the feature set size.  But we did not directly assessed how well the Laplace approximation affects the model performance. 

We have already mentioned that the proposed customized regularization model does not handle well high-dimensional  meta-features. This is because no regularization is applied to the meta-feature weights $\bm{\alpha}$, or other mechanism to control the complexity of this part of the model. As a result, the model fitting algorithm is not able to deal with large number of meta-features efficiently and stably. Careful filtering of the meta-features is then required in a high-dimensional meta-feature setting. Future work to regularize the meta-feature part of the model is a potential improvement to expand its ability of handling high dimensional meta-feature data.


% Research Topic 3
\chapter{\texorpdfstring{$L_0-$}{Lg}Regularized Regression with Correlated Features}
\label{cha:L0}

\section{Introduction}
In chapter \ref{cha:introduction},  we discussed the feature selection properties of the lasso, elastic net, and best subset selection. The latter is equivalent to $L_0$ constrained regression when data matrix $\bm{X}$ is orthogonal. In fact, the lassso and elastic net can be thought of as approximation methods to best subset selection. We see this in section \ref{sec:nonconvex}; in the spectrum of $L_q$ constrained regression, $q=0$ corresponds to best subset selection, $0<q<1$ to nonconvex regularization, $q=1$ to the lasso, and $q=2$ to ridge regression, which does not perfom selection. As the value of $q$ moving away from 0 to larger values, the estimated regression coefficients become more biased toward 0, and the resulting model becomes less sparse. Under orthogonality of the design matrix, best subset selection ($L_0$) yields unbiased coefficient estimates. Therefore, $L_0$ constrained regression would be generally the preferred model. However, in reality the features are not orthogonal, and exhibit some degree of correlations. In this situation, the $L_0$ constrained regression is an NP-hard problem \citep{huo2007stepwise}. 

Several approaches have been proposed for fitting the $L_0$-regularized linear regression. \cite{blumensath2008iterative} introduced an iterative hard thresholding algorithm, which is a proximal gradient-based method. Based on hard thresholding, several variants of the method have also been developed, such as proximal iterative hard thresholding \citep{zhang2019new}, and proximal alternating iterative hard thresholding \citep{yang2017proximal}. These methods share the main idea of applying the proximal operator univariately to obtain coordinate-wise solutions. However, these methods ignore the correlation structure of the data matrix $\bm{X}$, and might not work well with correlated features. We look at a novel algorithm to $L_0$ constrained least squares  and modify it to incorporate the covariance matrix $\bm{X}^T\bm{X}$ into account. The original plan for this work was to later extend $L_0$ constrained least squares to integrate meta-features. However, the approach did not work as expected and we turned to the methods in Chapters 2 and 3 instead. We include it here for completion and because some of the ideas may prove useful in the future.

\section[Proximal distance algorithm for \texorpdfstring{$L_0-$}{Lg}regularized regression]{Proximal distance algorithm for \texorpdfstring{$L_0-$}{Lg}regularized regression}
Proposed by \cite{keys2019proximal}, the proximal distance algorithm is a general method for solving a constrained optimization problem. It converts a constrained minimization problem into an unconstrained one, with a penalty on the distance to the constrain set. The constrained minimization problem is 
\begin{equation} \label{prox_con}
\begin{aligned}
    & \min_{x\in \mathbb{R}^p} f(x), \\
    & \text{subject to} \hspace{0.6cm} x\in C, 
\end{aligned}
\end{equation}
where $C$ is a closed set. This general constrained optimization problem can be turned into a penalized version (unconstrained)
\begin{equation} \label{prox_uncon}
   \min_{x\in \mathbb{R}^p} f(x)+\frac{\rho}{2}\text{dist}(x,C)^2,
\end{equation}
where the squared distance is defined as $\text{dist}(x,C)^2=\inf_{y\in C}\|x-y\|_2^2$, i.e. the square of the Euclidean distance of $x$ to the closed set $C$. The distance penalty function is nonnegative and vanishes precisely on $C$. As the penalty parameter $\rho$ tends to $\infty$, the distance of $x$ to set $C$ is penalized so strongly that it is close to 0, i.e., $x$ is in the set, which means the minimizer found by the unconstrained version \eqref{prox_uncon} is equivalent to the constrained one \eqref{prox_con}. To minimize \eqref{prox_uncon}, we again use a majorization-minimization (MM) algorithm similar to that used in chapter \ref{cha:xtunecox} to estimate the model hyperparameters. The majorization step, which forms an upper bound of $f(x)$ around the current iterate $x_n$, replaces the distance penalty function $\text{dist}(x,C)^2$ with the spherical quadratic $\|x-P_C(x_n)\|_2^2$, where $P_C(x_n)$ is the projection of the $n^{th}$ iterate $x_n$ onto C: the point in set C that attains the minimum distance to the point, $P_C(x_n)=\argmin_{x\in C}\|x_n-x\|_2$. Therefore, by the definition of projection, $\|x-P_C(x_n)\|_2^2\geq\text{dist}(x,C)^2$, and at current iterate $x_n$, the tangency condition, $\|x_n-P_C(x_n)\|_2^2=\text{dist}(x_n,C)^2$ of a majorization is satisfied. The minimization step is then the proximal map of the current projection
\begin{equation}
    x_{n+1}=\text{prox}_{\rho^{-1}f}(P_C(x_n))=\argmin_x f(x)+\frac{\rho}{2}\|x-P_C(x_n)\|_2^2.
\end{equation}
The MM principle guarantees that $x_{n+1}$ decreases the penalized loss.

For the $L_0$-regularized least squares problem, the constrained optimization form can be written as 
\begin{equation} \label{L0_con}
    \begin{aligned}
    \min_{\beta\in \mathbb{R}^p} \frac{1}{2}\|y-\bm{X}\beta\|_2^2, \\
    \text{subject to} \hspace{0.6cm} \beta \in S_k^p,
    \end{aligned}
\end{equation}
where $S_k^p$ is the set of vectors with at most $k$ nonzero entries out of $p$. The unconstrained form of \eqref{L0_con}, i.e., the penalized distance objective function is
\begin{equation} \label{L0_uncon}
    \min_{\beta\in \mathbb{R}^p} \frac{1}{2}\|y-\bm{X}\beta\|_2^2+\frac{\rho}{2}\text{dist}(\beta, S_k^p)^2.
\end{equation}
To use the majorization-minimization algorithm to solve problem \eqref{L0_uncon}, first we perform distance majorization
\begin{equation}
    \min_{\beta\in \mathbb{R}^p} \frac{1}{2}\|y-\bm{X}\beta\|_2^2+\frac{\rho}{2}\|\beta-P_{S_k^p}(\beta_n)\|_2^2.
\end{equation}
The minimization step is the proximal operator of the projection onto set $S_k^p$,
\begin{equation}
    \beta_{n+1}=\text{prox}_{\rho^{-1}\text{OLS}}(P_{S_k^p}(\beta_n))=(\bm{X}^T\bm{X}+\rho\bm{I})^{-1}(\bm{X}^Ty+\rho P_{S_k^p}(\beta_n)).
\end{equation}
We discussed earlier that the distance penaly parameter $\rho$ should be increased systematically until a constrained minimum is reached. In practice, starting $\rho_0=1$, and increasing its value through a sequence $\rho_n=\min(\alpha^n\rho_0, \rho_{\max})$ with  $\alpha$ slightly larger than 1 works well. The overall algorithm can be summarized as follow:
\begin{enumerate}
    \item Initialize $\beta_0=1$, $\rho_0=1$, $\text{dist}_0=\infty$, $\text{loss}_0=\infty$,
    \item For iteration counter $n$ from $1$ to max iteration:
    \begin{enumerate}
        \item Update $\beta_n=\text{prox}_{\rho^{-1}OLS}(P_{S_k^p}(\beta_{n-1}))$
        \item Current distance $\text{dist}_n=\|\beta_n-P_{S_k^p}(\beta_{n-1})\|_w^2$
        \item Current loss function vaue $\text{loss}_n=OLS(\beta_n)$
        \item If $|\text{dist}_n-\text{dixt}_{n-1}|<\text{tolerance}$ and $|\text{loss}_n-\text{loss}_{n-1}|<\text{tolerance}$:
        \begin{itemize}
            \item break
            \item return $P_{S_k^p}(\beta_n)$
        \end{itemize}
        else:
        \begin{itemize}
            \item $\text{dist}_{n-1}=\text{dist}_n$
            \item $\text{loss}_{n-1}=\text{loss}_n$
            \item increase $\rho$ by a small multiplier $\alpha$ every few iterations
        \end{itemize}
    \end{enumerate}
\end{enumerate}

\section{Incorporating the data covariance matrix}
The proximal distance algorithm uses the Euclidean distance as the distance metric. But this ignores the correlations between features. To incorporate the covariance matrix, we can use the weighted distance (Mahalanobis distance):
\[
d(x,y)=\sqrt{(x-y)^T\bm{W}^{-1}(x-y)},
\]
where, assuming the columns of $\bm{X}$ are mean-centered, $\bm{W}=\bm{X}^t\bm{X}$ is the data covariance matrix. With weighted distance majorization, the new update is 
\begin{equation}
    \beta_{n+1}=\text{prox}_{\rho^{-1}\text{OLS}}(P_{S_k^p}(\beta_n))=(\bm{X}^T\bm{X}+\rho\bm{W})^{-1}(\bm{X}^Ty+\rho\bm{W} P_{S_k^p}(\beta_n)).
\end{equation}
However, we need to project the current iterate to the $L_0$ constraint set, $P_{S_k^p}(\beta_n)$. With the standard Euclidean distance, the projection simply keeps the largest $k$ coordinates of $\beta$ and the remaining coordinates are set to 0. With the weighted distance, the projection problem is again NP-hard. We notice that if the weight matrix is diagonal, the projection keeps the $k$ coordinates with largest value of $w_j\beta_j^2$. Based on this observation, we propose the following approximation to the projection: for each coordinate $j$, compute the value $\frac{e_j^T\bm{W}\beta}{e_j^T\bm{W}e_j}$, which is the weighted projection onto axis $j$, and keep the axis projection value $\frac{(e_j^T\bm{W}\beta)^2}{e_j^T\bm{W}e_j}$, while setting the  rest of the coordinates to 0. This is similar to the projection method with the standard Euclidean distance but in a weighted inner-product space. 

\section{Simulation}
We simulated a data matrix $\bm{X}$ of dimension $100\times 200$, following a multivariate normal distribution, $N(0, \Sigma)$, where $\Sigma$ has an autoregressive(1) correlation structure with $\rho=0.1$, i.e., features are close to uncorrelated.  Regression coefficients $\bm{\beta}$ are also simulated with $N(0, \Sigma_\beta)$, where $\Sigma_\beta$ is autoregressive(1) with $\rho=0.6$. The variance components of $\bm{\beta}$ (diagonal elements of $\Sigma_\beta$) are set to be large for the first 10 elements, and small for the remaining 190. This setting makes the true underlying model sparse, i.e., the first 10 elements of $\bm{\beta}$ are non-zero, and the rest are all zeros. The data true predictive $R^2$ is fixed at $R^2=0.5$. Hyper-parameters of of the lasso ($\lambda$) and $L_0$ ($k$) are tuned using a simulated validation set. The experiment is repeated 100 times. The results (Table \ref{table:4.1}) showed that the validation $R^2$ of the lasso is higher than the validation $R^2$ for standard $L_0$ regression. In terms of feature selection, the true positive selection rate of lasso is also better than that of $L_0$ regression. However, this comes at the price of a higher false positive rate. We also looked at the bias of the first element of $\bm{\beta}$, where bias is the absolute value of the difference between the true value and the estimated value. The average bias of signal 1 of $L_0$ regression is much smaller than for the lasso.
\begin{table}[tbh]
    \centering
    \def\arraystretch{1.5}
    \begin{tabular}{|c|c|c|c|c|}
        \hline
         & \textbf{Validation $R^2$} & \textbf{Ture positive rate} & \textbf{False positive rate} & \textbf{Bias} (signal 1)  \\ 
        \specialrule{.1em}{.05em}{.05em}
        $L_0$ & 0.268 & 0.506 & 0.019 & 0.231 \\ 
        \hline
        lasso & 0.318 & 0.770 & 0.114 & 0.394 \\ 
        \hline
    \end{tabular}
    \caption{Prediction and feature selection comparison between lasso and $L_0$}
    \label{table:4.1}
\end{table}

We then looked at the 10 true signal coefficient estimates from a single run of the above experiment. Estimates from standard $L_0$ regression, our proposed weighted $L_0$ regression incorporating the data correlation structure, the lasso, along with the simulated true values are compared (Figure \ref{fig:L_0}). The standard $L_0$ selected 5 out of 10 true signals, while weighted $L_0$ and lasso both chose 8 of them. However, lasso estimates are heavily shrunk. Note that we used weighted distance metric in the weighted $L_0$ algorithm, but with the standard projection. 
\begin{figure}[tbh]
  \centering
  \includegraphics[width=\textwidth]{L_0}
  \caption{Comparison of signal estimates}
  \label{fig:L_0}
\end{figure}

The simulation results showed some potential for the weighted $L_0$ algorithm to improve feature selection in terms of accuracy and bias of the estimates. However, when we tried the proposed approximate solution to weighted projection in simulation, the results didn't show improvement in feature selection, and we haven't come up with a better solution to the weighted projection problem. 


% Conclusion and ongoing work
\chapter{Discussion and future work}
\label{cha:conclusion}

\section{Discussion on genomics data integration}
The purpose of this thesis is to develop novel modeling methods predicting health outcomes, based on genomics data. As the types and the volume of genomics data are huge thanks to advanced high-throughput sequencing technologies, as well as the ever-growing annotation databases, there is increased need to integrate multiple types of genomics data, annotation data into modeling process. Because related variables provide more information to the health outcomes of interest, and hence improve prediction performance. The traditional method is modeling one type of genomics data at a time, and combine the models in some form. As to utilizing annotation data, summary statistics, it is usually performed after modeling genomics data. This style of modeling different genomics data separately may ignore the interplay between them, the collective effect on the outcome. In this thesis, we introduced the concept meta-features, the features of the features, to incorporate external data. And we also introduced a meta-feature data matrix $\bm{Z}$ that systematically stores the external data. In the two methods developed in the thesis, chapter \ref{cha:xrnetcox}, \ref{cha:xtunecox}, we mainly discussed how to put annotation information into meta-feature matrix. That is, if we have $p$ genomic features, $q$ functional gene sets (meta-features), the meta-feature matrix $\bm{Z}$ will have dimension $p\times q$, each row represents one genomic feature and has values of 0 or 1 indicating whether this genomic feature belongs to a gene set (1 indicates it belongs to the gene set, and 0 not). However, the usage of meta-feature matrix does not limit to annotation data, in fact, it can accommodate many types of information. We discuss 2 situations to show the flexibility of putting external data into meta-feature matrix $\bm{Z}$.

\begin{itemize}
    \item There are 3 types genomics data, gene expressions, single nucleotide polymorphisms (SNPs), DNA methylation  to be integrated into the modeling process. The meta-feature matrix tells which genomic feature is SNP, gene expression, or methylation. Table \ref{table:d1} shows the indicator meta-feature matrix. For example, ILMN\_343291 is a microarray probe, gene expression; rs10853372 is a SNP locus. 
    \begin{table}[tbh]
    \centering
    \def\arraystretch{1.5}
    \begin{tabular}{|c|c|c|c|}
        \hline
         & \textbf{Gene expression} & \textbf{SNP} & \textbf{Methylation} \\ 
        \specialrule{.1em}{.05em}{.05em}
        ILMN\_343291 & 1 & 0 & 0 \\ 
        \hline
        rs10853372 & 0 & 1 & 0 \\ 
        \hline
        ILMN\_1651210 & 1 & 0 & 0 \\
        \hline
        463100A3 & 0 & 0 & 1 \\
        \hline
        \vdots & \vdots & \vdots & \vdots \\
    \end{tabular}
    \caption{Meta-feature matrix $\bm{Z}$ for multiple types of genomics data}
    \label{table:d1}
    \end{table}
    
    \item There are summary statistics from similar studies on the same set of genomic features. These statistics from meta-analysis can be highly informative. They include p-values, hazard ratios, and source of features. In table \ref{table:d2}, gene BAX has a p\_value 0.0006 associated with the outcome, hazard ratio is 0.7605, the reason being included in the model is from previous GWAS studies. This is a hybrid matrix holding continuous values and indicator values: continuous values like p\-values, hazard ratios gives importance of the features; indicator variable tells the reason why the feature is included. 
    \begin{table}[tbh]
    \centering
    \def\arraystretch{1.5}
    \begin{tabular}{|c|c|c|c|c|c}
        \hline
         & \textbf{p\_value} & \textbf{Hazard ratio} & \textbf{Literature} & \textbf{GWAS}  \\ 
        \specialrule{.1em}{.05em}{.05em}
        BAX & 0.0006 & 0.7605 & 0 & 1 & \dots \\ 
        \hline
        IL6 & 0.2611 & -0.2077 & 1 & 0 & \dots \\ 
        \hline
        LDHB & $8.78\times 10^{-6}$ & 0.0768 & 0 & 1 & \dots \\
        \hline
        \vdots & \vdots & \vdots & \vdots & \vdots & $\ddots$ \\
    \end{tabular}
    \caption{Meta-feature matrix $\bm{Z}$ for summary statistics}
    \label{table:d2}
    \end{table}
\end{itemize}

With the above examples, we are shown the flexibility of the meta-feature matrix housing external information. Through the meta-feature matrix, we can integrate multiple types of genomics data, genomic annotation data, summary statistics from similar studies, and so on. It is the heart of our modeling process to integrate extra information that might be useful to prediction.

\section{Discussion on high dimensionality of genomics data}
Most of the genomics data are high dimensional. The human genome contains approximately 3 billion base pairs, which reside in the 23 pairs of chromosomes within the nucleus of all our cells. Each chromosome contains hundreds to thousands of genes, which makes around 30,000 genes in the human genome. SNPs are common genetic variants happens roughly 1\% of the times among 3 billion base pairs in the human genome, so there are about 10 million SNPs. Over the years, many SNPs have been found. The number of genomics features, e.g., gene expressions, SNPs, DNA methylation, is huge by nature. However, the number of samples, especially in oncology setting, is small relative to the amount of genomics. For cancer patients, genomics data are obtained by tissue biopsy, which can be highly invasive, risky, costly. Some tumor locations are hard to access. And cancer patients are usually under serious health conditions which also prevent them from tissue biopsy. As a result, the number of patients with genomics data is limited. A typical data set in cancer genomics consists of hundreds to thousands subjects, and the number of genomic variables is tens to hundreds thousands, even millions. This makes the data ultra high dimensional ($p>>n$). We have already discussed in high dimensional setting, regularized regression as a linear model, is one of the better options, since every variable contributes a little to none effect to the outcome of interest. Intuitively, considering coming up a surface to classify samples in a high dimensional space, it is easy to use a hyperplane than a complicated non-linear surface. It is the opposite in a relatively low dimensional space, where non-linear pattern is favored. 

Recently, a novel technique, liquid biopsy which takes the human blood sample instead of tumor tissue, has been explored. It is less invasive, painful, easy to access, and can be performed in most health conditions. This could potentially generate more samples than tissue biopsies, and high dimensionality may no longer be an issue for genomics data. With both samples and features in high dimension, simple models such as linear models are not among the best any more. As machine learning has made great progress over the past years, there are better modeling choices. Reconsidering the claim that each genomic feature contribute a little to none effect to the outcome. While this is true, there could also be non-linear patterns like interactions between features that for example, over expression of one gene could cause other genes' expression change. Under limited samples, these complicated patterns are impossible to detect due to lack of information. However, with more samples, it is possible, and there are better models for exploring non-linear patterns. Gradient boosting machine is a tree-based method that specializing in detecting complicated interaction patterns. Deep neural network, with enough units and layers, can mimic any non-linear patterns. These two methods are hugely successful recently, because almost every data has some form of non-linear pattern. And we do see they need large amount of instances/samples to be successful. Liquid biopsy provides us the possibility to have large sample size while the number of genomic features remains large. Although it is not fully validated yet, in the near future, it can be expected to make impact on cancer genomics. As in cancer genomics, complicated non-linear pattern like square, higher order polynomials are hard to interpret the effects of genomics. Gene-gene interactions, gene-environment interactions, or even higher order interactions are of great interest. Therefore, in the case of having plenty of samples, developing methods to integrate genomics data for gradient boosting machine is a meaningful future work.

\section {Discussion on feature selection}
Up to now, the feature selection that we talked about is selecting a subset of the features that already included in the modeling process. Precisely speaking, this process is subset selection, which is to produce a parsimonious model that can reveal the underlying association between features and outcomes. Before further discussing subset selection, we first talk about feature selection, the process of selecting related features to build a predictive model. The predictive modeling process starts with laying out the question: what is the purpose of the predictive model, what type of outcome is to be predicted. Based on the scientific questions, we choose the data to collect that will give the most predictive power. In cancer genomics, data from tissue biopsy may contain different types of genomics data. It is not much of a problem to decide the types of data to be used based on the purpose, but rather, within one type of data, what subset of features to be included in the model. Because of the high dimensionality of data, one might want to narrow down the set of genomic features that have more predictive power. One common approach that is not recommended is to let the data decide the set of features by conducting pre-analysis based on the data. Because pre-screening the features based on the data and building predictive model afterwards is equivalent to using the data twice for modeling. It is problematic to train the model twice on different training sets. What can be done is utilizing knowledge from past studies, literature. But still, it is not a good idea to pre-screening features at all. The features that are considered not important may multivariately work with other features to be predictive to the outcome. Regularized regression has the ability to exclude unimportant features by shrink them to 0; gradient boosted trees can ignore those features by not even using them as tree nodes. It is better to let the model decide how each feature contributes to prediction.

Now in terms of subset selection in regularized regression, the two methods developed in this thesis both used most common sparse regularization technique, the lasso and the elastic net. Because they are both convex, and have stable and efficient algorithms. Moreover, as we discussed in section \ref{sec:sparse}, the lasso subset selection is consistent under certain conditions, and the elastic net can complement the lasso to deal with group correlated features. However, as convex approximations to best subset selection, they both produce estimators biased toward 0, and there exist some conditions in which their subset selection are not consistent. Discussed in section \ref{sec:nonconvex}, nonconvex regularizations, which are more closer approximations to best subset selection, produce more parsimonious model, and the estimators are less biased toward 0. The 2 nonconvex regularizations that are discussed, the SCAD and the MCP, along with the adaptive lasso, approximation to $L_q (0<q<1)$ regularization, enjoy oracle property; namely, when the true estimators have some zero components, they are estimated as 0 with probability tending to 1, and the nonzero components are estimated as well when the correct submodel is known \citep{fan2001variable}. This property improves the model accuracy compared to the lasso and the elastic net. As the 3 regularization techniques enjoy coordinate descent algorithm, they are natual extensions to the two methods developed in this thesis, for the purpose of better subset selection, in terms of both accuracy and unbiased estimator. 

\section{Discussion on statistical inference}

% Using single-space for reference list.
\begin{singlespace}
% Bibliography
\phantomsection
\addcontentsline{toc}{chapter}{References}%
\markboth{References}{References}%
% If you use BibLaTeX
%\printbibliography[title=References]
% If you use BibTeX
% \bibliographystyle{plain}
\bibliography{references}
\end{singlespace}

% Appendices
\phantomsection
\addcontentsline{toc}{chapter}{Appendices}%
\markboth{Appendices}{Appendices}%
\chapter*{Appendices}
\renewcommand\thesection{\Alph{section}}
\renewcommand*{\thesubsection}{\Alph{section}.\arabic{subsection}}
\begingroup
\numberwithin{equation}{section}
% Appendix source files
\section{Appendix for chapter 2}

\subsection{Computation of diagonal elements of weight matrix}
\label{a.1}
Diagonal elements of weight matrix $\bm{W}$, the Hessian of log of Cox'x partial likelihood function, has the form 
\begin{displaymath}
w_i=\sum_{k\in C_i}\frac{d_k e^{\tilde{\eta}_i}}{\sum_{j\in R_i}e^{\tilde{\eta}_j}}-\sum_{k\in C_i}\frac{d_k (e^{\tilde{\eta}_i})^2}{(\sum_{j\in R_i}e^{\tilde{\eta}_j})^2}. 
\end{displaymath}
The two sums $\sum_{k\in C_i}$ and $\sum_{j\in R_i}$ both have $n$ elements, hence it is a $O(n^2)$ computation. However, if we notice the difference between $R_k$ and $R_{k+1}$ is the observations that are in $R_k$ but not in $R_{k+1}$, i.e., $\{j:t_k\leq y_j < t_{k+1}\}$, provided that the observed times $\bm{y}$ are sorted in non-decreasing order, then $\sum_{j\in R_k}e^{\tilde{\eta}_j}$ can be calculated as cumulative sums:
\begin{displaymath}
\sum_{j\in R_k} e^{\tilde{\eta}_j} =\sum_{j\in R_{k+1}} e^{\tilde{\eta}_j}+ \sum_{j\in R_k \& j\notin R_{k+1}} e^{\tilde{\eta}_j}.
\end{displaymath}
The same cumulative sum idea can be applied to $\sum_{k\in C_i}$: 
\begin{align*}
    \sum_{k\in C_{i+1}}\frac{d_k e^{\tilde{\eta}_i}}{\sum_{j\in R_i}e^{\tilde{\eta}_j}}&=\sum_{k\in C_i}\frac{d_k e^{\tilde{\eta}_i}}{\sum_{j\in R_i}e^{\tilde{\eta}_j}}+\sum_{k\in C_i\&k\notin c_{i+1}}\frac{d_k e^{\tilde{\eta}_i}}{\sum_{j\in R_i}e^{\tilde{\eta}_j}}, \\
    \sum_{k\in C_{i+1}}\frac{d_k (e^{\tilde{\eta}_i})^2}{(\sum_{j\in R_i}e^{\tilde{\eta}_j})^2}&=\sum_{k\in C_i}\frac{d_k (e^{\tilde{\eta}_i})^2}{(\sum_{j\in R_i}e^{\tilde{\eta}_j})^2}+ \sum_{k\in C_i\&k\notin c_{i+1}}\frac{d_k (e^{\tilde{\eta}_i})^2}{(\sum_{j\in R_i}e^{\tilde{\eta}_j})^2}.
\end{align*}
The equations above only calculate the sums once, and add at each sample index, which brings the computation cost down to linear time, $O(n)$. Considering sorting observed times as a data pre-processing procedure, the overall computation time for the weights is $O(n\log n)$.

\subsection{Solve regularized weighted least squares with cyclic coordinate descent}
\label{a.2}
To solve regularized weighted least squares, equation \eqref{eq2.9}, we first compute the gradient at current estimates of $(\tilde{\bm{\gamma}}, \tilde{\bm{\alpha}})$. Let $\gamma_j$ be the $j^{th}$ coordinate of $\bm{\gamma}$, $1\leq j\leq p$; $\alpha_k$ be the $k^{th}$ coordinate of $\bm{\alpha}$, $1\leq k\leq q$. The gradient of equation \ref{eq2.9} with respect to $\gamma_j$ is 
\begin{displaymath}
-\frac{1}{n} \sum_{i=1}^n w_ix_{ij}(y'_i-\bm{\gamma}^T\bm{x}_i-\bm{\alpha}^T(
\bm{xz})_i) + \lambda_1\gamma_j.
\end{displaymath}
Setting the gradient with respect to $\gamma_j$ to 0, the coordinate-wise update for $\gamma_j$ has the form 
\begin{displaymath}
\gamma_j = \frac{\frac{1}{n}\sum_{i=1}^n w_ix_{ij}r_i^{(j)}}{\frac{1}{n}\sum_{i=1}^nw_ix_{ij}^2+\lambda_1},
\end{displaymath}
where $r_i^{(j)}=y'_i-\sum_{l\neq j}\tilde{\gamma}_lx_{il}-\tilde{\bm{\alpha}}^T(\bm{xz})_i$, is the partial residual excluding the contribution of $x_{ij}$. As for $\alpha_k$, if $\tilde{\alpha}_k>0$, the gradient of equation \eqref{eq2.9} with respect to $\alpha_k$ is 
\begin{displaymath}
-\frac{1}{n}\sum_{i=1}^n w_i(xz)_{ik}(y'_i-\bm{\gamma}^T\bm{x}_i-\bm{\alpha}^T(
\bm{xz})_i) + \lambda_2.
\end{displaymath}
A similar expression exists if $\tilde{\alpha}_k<0$, and $\tilde{\alpha}_k=0$ is treated separately. Setting the gradient with respect to $\alpha_k$ to 0, the coordinate-wise update for $\alpha_k$ has the form 
\begin{displaymath}
\alpha_k = \frac{S(\frac{1}{n}\sum_{i=1}^n w_i(xz)_{ik}s_i^{(k)}, \lambda_2)}{\frac{1}{n}\sum_{i=1}^n w_i(xz)_{ik}^2},
\end{displaymath}
where $s_i^{(k)}=y'_i-\tilde{\bm{\gamma}}^T\bm{x}_i-\sum_{l\neq k}\tilde{\alpha}_l(xz)_{il}$, is the partial residual excluding the contribution of $(xz)_{ik}$. $S(z,\lambda)$ is the soft-thresholding operator:
\begin{equation*}
    \text{sign}(z)(|z|-\lambda)_+ = 
        \begin{cases}
            z-\lambda & \text{if $z>0$ and $\lambda<|z|$}\\
            z+\lambda & \text{if $z<0$ and $\lambda<|z|$}\\
            0 & \text{if $\lambda \geq |z|$}
        \end{cases}       
\end{equation*}


\subsection{Example codes for R package `xrnet'}
\label{a.3}
The regularized hierarchical model, chapter \ref{cha:xrnetcox}, is implemented in R package `xrnet'. The package is in CRAN and can be downloaded as any other R packages. Up to the time of this thesis being written, the CRAN version of `xrnet' implements with respect to continuous and binary outcomes. The survival module is in development branch of github repository. It can be installed with the command

\begin{lstlisting}[language=R]
devtools::install_github("USCbiostats/xrnet",ref="development")
\end{lstlisting}

We give example codes for using 'xrnet' with respect to survival outcomes. As an minimum example, function `xrnet' performs the regularized hierarchical model, data and the type of model to be performed must be provided.

\begin{lstlisting}[language=R]
library(xrnet)
fit = xrnet(x=x, y=y, external=z, family="cox")
\end{lstlisting}
Argument $x$ is the data matrix, $y$ is the outcome, external is meta-feature data matrix. If external data is not provided, a standard regularized regression is performed. Family=``cox'' indicates the type of model, in this case, Cox's proportional hazards model. To specify the regularization type and regularization path, the helper function `define\_penalty' can be used. 
\begin{itemize}
    \item Regularization type
    \begin{itemize}
        \item 0 := ridge
        \item 1 := lasso
        \item (0,1) := elastic net
    \end{itemize}
    \item Regularization path
    \begin{itemize}
        \item Number of the sequence for each of $\lambda_1$ or $\lambda_2$
        \item Ratio $\lambda_{\min}/\lambda_{\max}$
    \end{itemize}
    \item User defined sequence of regularization parameters
\end{itemize}
The arguments `penalty\_main' and `penalty\_external' are used to specify the above regularization options to the features in data matrix $\bm{X}$ and to the meta-features in $\bm{X}$. For example, we apply ridge to the features, and lasso to the meta-features. Each of penalty parameter sequences has 20 values. The codes are as follow

\begin{lstlisting}[language=R]
fit = xrnet(x=x, y=y, external=z, family="cox",
            penalty_main=define_penalty(0, num_penalty=20),
            penalty_external=define_penalty(1, num_penalty=20))
\end{lstlisting}
Help function `define\_ridge', `define\_enet', `define\_lasso' are available to directly specify the type of regularization. 

In order to tune the hyperparameters, $\lambda_1, \lambda_2$, cross-validation is used. In `xrnet' package, function `tune\_xrnet' is for cross-validation

\begin{lstlisting}[language=R]
cvfit = tune_xrnet(x=x.train, y=y.train, external=z, 
                   family="cox",
                   penalty_main=define_ridge(), 
                   penalty_external=define_lasso(),
                   loss="c-index", nfolds=5)
\end{lstlisting}
The example code shows that it performs 5 fold cross-validation (nfolds=5), and the validation metric is C-index (Harrell's concordance index). The folds can also be specified by user with argument `foldid'. To predict and evaluate prediction performance on a hold out test set, with cross-validated model, use the following codes

\begin{lstlisting}[language=R]
library(glmnet) # for function Cindex
pred = predict(cvfit, newdata=x.test)
test_cindex = Cindex(pred, y.test)
coefficient = predict(cvfit, type="coefficients")
\end{lstlisting}

\subsection{More simulation results}
We conducted 6 experiments for the regularized hierarchical Cox's regression (`xrnet'). The base case scenario is meta-feature signal-noise ratio $\text{SNR}=2$, sample size $N=100$, number of features $p=200$, number of meta-features $q=50$, theoretical/population concordance index 0.8, data matrix $\bm{X}$ correlation $\rho=0.5$. In every experiment, we vary one of the 6 parameters and hold others fixed. Prediction performance, and meta-feature selection accuracy are shown for each experiment.
\begin{figure}[H]
    \includegraphics[width=\textwidth]{C12}
    \caption{Simulation: `xrnet' prediction performance (i)}
    \label{fig:C12}
\end{figure} 

\begin{figure}[H]
    \includegraphics[width=\textwidth]{C34}
    \caption{Simulation: `xrnet' prediction performance (ii)}
    \label{fig:C34}
\end{figure} 

\begin{figure}[H]
    \includegraphics[width=\textwidth]{C56}
    \caption{Simulation: `xrnet' prediction performance (iii)}
    \label{fig:C56}
\end{figure} 

\begin{figure}[H]
    \includegraphics[width=\textwidth]{acc12}
    \caption{Simulation: `xrnet' meta-feature selection (i)}
    \label{fig:acc12}
\end{figure} 

\begin{figure}[H]
    \includegraphics[width=\textwidth]{acc34}
    \caption{Simulation: `xrnet' meta-feature selection (ii)}
    \label{fig:acc34}
\end{figure} 

\begin{figure}[H]
    \includegraphics[width=\textwidth]{acc56}
    \caption{Simulation: `xrnet' meta-feature selection (iii)}
    \label{fig:acc56}
\end{figure} 

\section{Appendix for chapter 3}
\subsection{More simulation results}
We conducted 4 experiments for the meta-feature guided regularized regression model. The base case scenario is high meta-feature informativeness, sample size $N=100$, number of features $p=200$, number of meta-features $q=10$. In every experiment, we vary one of the 4 parameters and hold others fixed. We have seen the prediction performance results (Figure \ref{fig:sim21}), here we show feature selection accuracy, and number of features selected.

\begin{figure}[H]
    \includegraphics[width=\textwidth]{acc212}
    \caption{Simulation: meta guided feature selection accuracy (i)}
    \label{fig:acc212}
\end{figure} 

\begin{figure}[H]
    \includegraphics[width=\textwidth]{acc234}
    \caption{Simulation: meta guided feature selection accuracy (ii)}
    \label{fig:acc234}
\end{figure} 

\begin{figure}[H]
    \includegraphics[width=\textwidth]{sel212}
    \caption{Simulation: meta guided number of selected features (i)}
    \label{fig:sel212}
\end{figure} 

\begin{figure}[H]
    \includegraphics[width=\textwidth]{sel234}
    \caption{Simulation: meta guided number of selected features (ii)}
    \label{fig:sel234}
\end{figure} 


\endgroup

% In case your dissertation has multiple volumes.
% \addvolumecontents{thesis_part2}
% \addvolumecontents{thesis_part3}
% \addvolumecontents[lof]{thesis_part2}

\end{document}
